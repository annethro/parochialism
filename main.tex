%                                                                 aa.dem
% AA vers. 9.1, LaTeX class for Astronomy & Astrophysics
% demonstration file
%                                                       (c) EDP Sciences
%-----------------------------------------------------------------------
%
%\documentclass[referee]{aa} % for a referee version
%\documentclass[onecolumn]{aa} % for a paper on 1 column  
%\documentclass[longauth]{aa} % for the long lists of affiliations 
%\documentclass[letter]{aa} % for the letters 
%\documentclass[bibyear]{aa} % if the references are not structured 
%                              according to the author-year natbib style

%
\documentclass[bibauthoryear]{aa}  
\usepackage{blindtext}
\usepackage[inline]{enumitem}
\usepackage{xcolor}
\usepackage{graphicx}
%%%%%%%%%%%%%%%%%%%%%%%%%%%%%%%%%%%%%%%%
\usepackage{txfonts}
%%%%%%%%%%%%%%%%%%%%%%%%%%%%%%%%%%%%%%%%
\usepackage{hyperref}
% To add links in your PDF file, use the package "hyperref"
% with options according to your LaTeX or PDFLaTeX drivers.
%



\usepackage{epstopdf}
\usepackage{xcolor}
\usepackage{rotating}

\hypersetup{
    colorlinks,
    linkcolor={blue!50!black},
    citecolor={blue!50!black},
    urlcolor={blue!80!black}
}
\usepackage{pgfplotstable}
\newcommand{\sbx}[2][c]{%
  \begin{tabular}[#1]{@{}c@{}}#2\end{tabular}}
  
\usepackage{graphicx}
\usepackage{epstopdf}
%% The amssymb package provides various useful mathematical symbols
\usepackage{amssymb}
%% The amsthm package provides extended theorem environments
 \usepackage{amsmath}
 \usepackage{enumitem}
 \usepackage{subcaption}

%% make sure you have the nature.cls and naturemag.bst files where
%% LaTeX can find them

%%%% *** Do not adjust lengths that control margins, column widths, etc. ***
\usepackage{amsfonts}
\usepackage[utf8]{inputenc}
\usepackage{url}
\usepackage{array,booktabs}
\newcolumntype{L}{@{}>{\kern\tabcolsep}l<{\kern\tabcolsep}}
\usepackage{colortbl}
\usepackage{xcolor}
\usepackage{graphicx}
\usepackage{caption}
\usepackage{subcaption}
\usepackage{amsmath}
\usepackage{amssymb}
\usepackage{rotating}
\usepackage{multirow}
\usepackage{arydshln}

\setlength{\tabcolsep}{3pt}

 \usepackage{graphicx,rotating,booktabs}
 \usepackage{framed}
\definecolor{shadecolor}{cmyk}{.04,.04,.12,.08}

\usepackage{lipsum}

\usepackage{multicol}

\setlength{\columnsep}{0.6cm}



\newcounter{framecnt}
\newenvironment{frameenv}[1]
    {\begin{figure}[tb]
    \refstepcounter{framecnt}
    \begin{shaded}
    \renewcommand{\theHfigure}{cont.\arabic{framecnt}}
  
    \textbf{\centerline{Box \arabic{framecnt} --- #1}}
         }
    {\end{shaded}\end{figure}
    }
\usepackage[left,modulo]{lineno}
\usepackage[a4paper,includeheadfoot,margin=2.25cm]{geometry}


\begin{document} 


   \title{Parochial altruism in humans may be universally possible, but is not universally present}

\author{%%%% Author details
Anne C. Pisor$^{1,2}$ and Cody T. Ross$^{2}$
}
%%%%%%%%% Insert author address here
\institute{$^{1}$ Department of Anthropology, Washington State University.
\\
$^{2}$ Max Planck Institute for Evolutionary Anthropology. Dept. of Human Behavior, Ecology and Culture. Germany.
}

   \date{}

% \abstract{}{}{}{}{} 
% 5 {} token are mandatory
 
  \abstract
 {
Parochial altruism, or in-group favoritism at out-group expense, is popularly believed to be a universal in humans---something that characterizes all societies. However, the empirical literature points to considerable variability in the expression of parochial preferences. We argue that greater emphasis should be placed on understanding this variation and how it can be impacted both by research design and by cross-cultural variation in the constraints on and incentives for inter-group tolerance. In what follows, we draw on two illustrative case studies conducted in Colombia and Bolivia to discuss the flexible nature of parochial altruism in humans. By deploying multiple methods to measure parochialism, in multiple communities with members from the same ethnic and religious groups, we show that the degree of parochial altruism expressed in these communities can be linked to the constraints and incentives faced by individuals in specific contexts, but also that it can be influenced by methodological choices. Our case studies both highlight how our methods may impact our inferences and suggest that while parochial altruism is universally possible in humans, it is not universally present across communities, across individuals, or even within the same individual across time. We close by offering concrete considerations for how researchers can better capture real-world variability in parochial altruism by minimizing the noise caused by the methods they choose.
 }

   \keywords{parochial altruism, parochialism, cooperation, intergroup relationships, intergroup conflict, human sociality
               }
               
               \titlerunning{~}
\authorrunning{~  }%

   \maketitle
%
%-------------------------------------------------------------------

% Word limit for BBS: approx 12,000 (without references)
% Current word count on August 22: 14,484


\section{Introduction}\label{firstbit}
Parochial altruism\footnote{\emph{Parochial altruism} falls under the umbrella of \emph{parochialism} along with \emph{xenophobia}; see \citet{hruschka2013economic} for discussion.}, or in-group favoritism at out-group expense \citep{choi2007coevolution}, is assumed to be a central feature of human behavior, both by social scientists and by the general public; sometimes it is even referred to as a human universal \citep{greene2013moral}. But is it? Existing work has found mixed support for parochial altruism in contemporary populations \citep{Rusch2014}, a finding which may be at least partially due to the contingent and flexible nature of parochial attitudes in humans. Parochial attitudes can be expected to vary as a function of cultural institutions, an individual's state (e.g., their current wealth), incentives for competition over resources,  past exposure to out-group members, or a combination of these and other factors \citep{pisor2019evolution}. This mixed support may also be partially due to the decoupling of in-group favoritism from the generation of out-group costs---a possibility hypothesized by several researchers \citep{purzycki2019identity, hruschka2013economic, yamagishi2016parochial, brewer2006evolutionary, schaub2017threat, cashdan2001ethnocentrism}. Moreover, varied evidence for parochial altruism may also stem from variation in methods used to collect it \citep{Pisor2020}.



In this paper, we discuss these potential sources of variation in parochial attitudes across individuals and communities, focusing especially on the role of methodology. Drawing on our own field data from Colombia and Bolivia, we show that the level of  parochial altruism  exhibited by individuals reflects both the constraints and incentives they face when it comes to interacting with other groups, as well as the ways in which we as researchers measure parochial altruism, including whether we make in-group favoritism independent of, or contingent on, out-group cost generation. The flexibility we capture in parochial altruism underscores the fact that though parochial altruism is likely a human universal in that it is universally \textit{possible}, it is not universally present across communities or even across time within communities.


\subsection{The study of parochial altruism}\label{onepointone}

While the study of intergroup conflict in humans has been a long-time feature of research in the social sciences---including psychology \citep[e.g.,][]{tajfel1982social, yamagishi2016parochial}, sociology \citep[e.g.,][]{gluckman1960tribalism}, and anthropology \citep{Vayda1961}---the study of parochial altruism itself is comparatively new. The influence of this concept over the last two decades was propagated in large part by high-profile theoretical publications by Sam Bowles and colleagues \citep{choi2007coevolution, bowles2003origins, bowles2004persistent}. According to this work, parochial altruism may have emerged, at least partially, as a product of genic selection: the ancestors of contemporary \textit{Homo sapiens} lived in a foraging ecology that required within-community food sharing to smooth the risk inherent in hunting, but access to resources, both hunted and otherwise, was a source of competition between groups \citep{choi2007coevolution}. Because of the importance of within-group relationships to both manage the risk of hunting failures and to defend resources against other groups, parochial altruism may have been favored in humans. In other words, groups may serve as containers for cooperation \citep{boydricherson1985}. However, \citet{bowles2003origins} predict that the degree to which parochial altruism is observed in a particular population should reflect the local costs and benefits of intergroup competition, a distinction often absent in the citing literature. For example, \citet{bowles2004persistent} demonstrate that parochially altruistic groups can exist within larger networks in which groups are interdependent because of opportunities for trade. (However, see \citet{yamagishi2011trust} and \citet{yamagishi2016parochial} for an alternative interpretation of social networks with this structure.)

 The premise of parochial altruism is that within-group cooperation can generate group-level benefits sufficient to off-set individual-level costs---that is, that the fitness benefits to individuals in the group outweigh the costs they pay. % however, the size of the group that is receiving in-group favoritism and generating out-group costs is somewhat inconsistent, both within and across many publications on parochial altruism. 
	In other words, natural selection can favor costly behavior that promotes the fitness of in-group members for a group of any size, as long as there are coincident fitness interests across members of that group \citep{richerson2008not}. Despite the implicit assumption in the literature that a group is a hunter-gatherer camp, within which fitness interests are mostly coincident across individuals\footnote{It is true that most needs can be met in the context of a residential camp \citep[e.g.,][]{fafchamps2007formation}, but certainly not all needs \citep{hill2014hunter, layton2012antiquity}.} %Cody: this was an addition that didn't quite click for me. I tried to reword to smooth it out, but I might have missed your meaning. As stated, it conflicts with the statement at the end of the paragraph that most folks are talking about ethnicities when they're talking about parochial altruism, which was my read of the literature. Did you have a different read?
 \citep{bowles2003origins}, authors refer to the in-group as ``demes''---groups large enough to include strangers \citep[see][for a discussion]{brewer2006evolutionary}---or as ``ethnicities'' \citep{choi2007coevolution}. Indeed, many articles settle on the in-group/out-group distinction as an ethnic boundary \citep{mcelreath2003shared}. ``Ethnicity''\footnote{This definition is limited, not reflecting the usage of the word in other disciplines or outside of academia; see \citep{jenkins1994rethinking} for a discussion.} in this context refers to shared cultural institutions \citep{barth1956ecologic, barth1998ethnic}, or ways of doing things \citep{north1991}; institutions involve norms, rules, or laws that help individuals coordinate to increase their benefits, gained from things like economic transactions or public works \citep{glowacki2020}. In short, because of shared cultural institutions that may generate coincident fitness interests, ethnicity is often taken as the default group in the parochial altruism literature unless the definition of ``group'' is explicitly extended by the author to include other categories of people that share cultural institutions---e.g., to include nations \citep{greene2013moral} or religions \citep{purzycki2016moralistic}.
	
Theoretical and empirical work on parochial altruism has cross-pollinated with related literatures, especially those focused on between-group competition in humans. The parochial altruism literature is perhaps most closely related to the strong reciprocity literature, which posits that the moral emotions present in humans ensure group-beneficial acts, including costly punishment that enforces group-beneficial norms \citep{fehr2002strong, fehr2003strong, gintis2008strong}. However, the parochial altruism literature has also influenced (and been influenced by) the literatures investigating the origins of human warfare \citep{glowacki2017evolutionary, wrangham2012intergroup, zefferman2015evolutionary}, identity fusion with in-group members \citep[i.e., conflating the fitness interests of the self and others;][]{swann2012group,  purzycki2019identity}, and cultural group selection \citep{richerson2016cultural}. The inclusion of the tenets of parochial altruism in popular books \citep{seabright2004company, greene2013moral} and policy recommendations \citep{choi2019parochialism, waring2015} has further broadened the scope of their influence.

The incidence of tolerant behavior toward out-group members, and even of social relationships spanning group boundaries, is often not addressed by these literatures; when it is, it is often as an afterthought, as an example of how human tendencies toward parochial altruism can be overridden \citep{choi2007coevolution, glowacki2017evolutionary}. Authors disagree as to how this ``override'' may occur \citep{pisor2019evolution}. Most do not posit a genetic adaptation for inter-group tolerant behavior, but rather suggest one of the following possible explanations. First, cultural institutions may enforce tolerant behavior toward members of other groups \citep{fearon1996explaining, fry2018evolutionary}, though there is disagreement about whether such institutions appear, persist, and spread primarily because they generate individual-level benefits or group-level benefits \citep[see][for a useful discussion]{purzycki2020institutions}. Second, as an individual is exposed to more out-group members, they may develop additional dyadic loyalties and become less willing to favor individuals of their own ethnic group over those of another ethnic group; there are various explanations for why this happens, from identity fusion to the strategic building of social capital \citep{brewer1976ethnocentrism, beck2006cosmopolitan, buchan2009globalization, fukuyama2001social, hruschka2013economic, mau2008cosmopolitan, singer2011expanding}. Third, when an individual has their basic needs met, they may be more likely to consider the well-being of out-group individuals---because they can afford to \citep{hruschka2014impartial, silva2014cooperation}. However, with few exceptions \citep{hruschka2013economic, vardy2019property}, those studying parochial altruism tend not to indicate whether one of these by itself---institutions, exposure, or basic needs---is sufficient to explain the variation in parochial altruism observed in humans.

Despite a lack of emphasis on the flexibility of parochial altruism in the literature, parochial altruism has been found, and not found, all over the world \citep{Rusch2014, Baldassarri1183}. There are a few possible explanations for this variability. First, per the above, it may be that some groups do not have institutions promoting tolerance of out-group members, have limited past exposure to out-group members, or cannot meet their own basic needs and thus cannot afford to care about out-group members. Second, it may be that despite influential theories positing a link between in-group favoritism and out-group cost generation \citep{bowles2003origins}, these phenomena are actually decoupled; compelling data indicate that high levels of in-group altruism can occur without out-group cost generation \citep{purzycki2019identity, hruschka2013economic, yamagishi2016parochial, brewer2006evolutionary, schaub2017threat, cashdan2001ethnocentrism, Rusch2014}. Third, it may be that parochialism is better detected using some methods than others---in effect, that researchers can purposefully, or more commonly, inadvertently activate more parochial or more tolerant attitudes towards outsiders through the methods they use \citep{Pisor2020}. We evaluate all three possibilities over the course of this paper, but the third is our central focus: there is likely real variation in the extent of parochial altruism within and between different contemporary human populations, but our ability to measure this variation is impacted by the methods we use.

\subsection{Is parochial altruism in the method?}\label{inthemethod}

The possibility that methodology can cloud our interpretation of real social phenomena is not unique to the study of parochial altruism. Consider the study of generalized trust---trust that a stranger encountered on the street will not cause you harm \citep{yamagishi2011trust}. When researchers study trust and attempt to make inferences about generalized trust at the state or country level, they often rely on a question found on the General Social Survey, the European Values Survey, and the World Values Survey: ``Generally speaking, would you say that most people can be trusted or that you need to be very careful in dealing with people?'' This question was originally from a ``faith in people'' scale and was incorporated into these large-scale surveys with minimal vetting \citep{miller2003surveys}. The first problem with the question is that it conflates trust with caution \citep{miller2003surveys}, creating confusion among respondents \citep{nannestad2008have}. Second, it lacks both internal and external validity \citep{loewenstein1999experimental}: it lacks internal validity because it does not replicate within the research context using other methods \citep{glaeser2000measuring}, and it lacks external validity because it bears little relation to participants' trusting behavior in ``the real world,'' outside of the reearch context \citep{nannestad2008have}. In other words, a  mainstream measure of generalized trust seems not to replicate well using other methods, nor to reflect the reality of people's behavior. Drawing on the lessons learned from the study of generalized trust, we can ask whether the methods used to study parochial altruism likewise: (i) conflate different research questions, and (ii) lack either external validity (that is, bearing on the real world) and/or internal validity (replicability across different measures). 

Parochial altruism is most commonly studied using experimental economic games. Economic games were imported from experimental economics to anthropology in the early 2000s by Joe Henrich and colleagues \citep{henrich2001search} and became a commonly used means of quantifying cooperative intent among subsistence populations from around the globe \citep{henrich2005economic, ensminger2014experimenting}. Games vary in design, but always involve a decision of whether and how much of a particular currency (usually money) to allocate to others---often to one anonymous, same-community recipient \citep[see][for an overview]{Pisor2020}. For example, in a classical game called the Dictator Game, the decider makes an offer and the recipient has no choice but to accept it; in the Ultimatum Game, the recipient can accept or reject a decider's offer; and in the Third-Party Punishment Game, a third individual learns the results of a Dictator Game and, at a cost to themselves, can punish (or not) the decider for their offer. Because these games involve anonymous giving to a same-community individual and anonymous costly punishment, they have been interpreted as indicative of strong reciprocity---the anonymous giving as indicating willingness to give to any in-group member, and the costly punishment as indicating willingness to pay a personal cost to benefit any in-group member\citep{marlowe2008more} \citep[see][for a review]{guala2012reciprocity}.

The problem with these classical economic games is that they can conflate different research questions, raising concerns about internal and external validity \citep{Pisor2020, Naar2020}, just as was the case in the study of generalized trust. First, classical games were primarily designed to study bargaining \citep{Camerer1995}, not cooperation with in-group members at the cost of out-group members---parochial altruism \citep[e.g.,][]{yamagishi2016parochial}---or, in the case of strong reciprocity, costly punishment \citep{hagen2006game, guala2012reciprocity}. For example, models of parochial altruism implicitly posit that cooperation is zero-sum: effort invested into cooperation with in-group members comes at the expense of out-group members. This trade-off is not part of the design of the Dictator, Ultimatum, or Third-Party Punishment games \citep[see][for relevant disucssion]{Gil-white2004}. Second, researchers have raised concerns that games played with anonymous, same-community others often lack external and internal validity. Game behavior often has little association with behavior in the real world \citep{gurven2008collective, winking2013natural}---at least, when the task is not framed in terms of a particular real-world context \citep{cronk2007influence, hagen2006game, lesorogol2005experiments, lesorogol2007bringing, pisor2012importing, lightner2017}---and behavior appears to be consistent across different versions of economic games only if participants are consistent in the way they think about them, e.g., framing all games in terms of opportunities for cooperation \citep{yamagishi2013behavioral}. Because of the disconnect between the theoretical question asked and the classical games used, as well as persistent questions about the external and internal validity of these methods, there are lingering discussions about what is and is not known about parochial altruism in humans \citep{Rusch2014,yamagishi2016parochial}.

\subsection{The present contribution}

We (ACP and CTR) came to the study of parochial altruism wanting to better understand its \textit{flexibility} across individuals and across communities. CTR was interested in whether parochialism constrains social networks and resource transfer networks more intensely at ethnic boundaries. ACP wanted to know whether the need for non-local resource access would relax parochial attitudes, leading individuals to express preferences for cooperative social relationships spanning ethnic or religious boundaries. We both decided that to document this flexibility, we needed methods other than classical economic games for the reasons described above. We each used a combination of ethnography and experimental\footnote{We refer to the economic games and ACP's choice task as ``experiments''. This word usage is consistent with many areas of economics and psychology where economic games are treated as ``experiments'' \citep{guala2005methodology, Cattell1988}. Nevertheless, an independent variable is  not always manipulated in these games.  Such a manipulation is generally possible, however, if desired.} methods designed to either capture the real world  or to capture the ``private world''---how individuals would behave if they could \citep{Pisor2020}. We did not intend them to, but our respective studies provide illustrative examples of the discussion above. We find that subtle features of methodological design affect both our ability to identify real-world parochialism (or lack thereof) using experimental methods and to replicate such findings (or lack thereof) \textit{across} experimental methods.  %Further, through a combination of our experimental approaches and ethnography, we were able to (i) document differences across individuals and communities with respect to parochial attitudes---including differences due to resource competition and past experiences with out-groups, (ii) illustrate that preferences for long-distance relationships might be mistakenly detected as preferences for inter-ethnic relationships using some research protocols, and (iii) document a disconnect between preferences for in-group favoritism and the generation of out-group costs.

In what follows, we draw on two illustrative case studies---one from rural Colombia (CTR) and one from rural Bolivia (ACP)---to discuss the flexible nature of parochial altruism in humans. We show that the degree of parochial altruism in these communities can be reliably predicted by the constraints and incentives faced by individuals in a given context, but can also be influenced by methodological choices, as evidenced by interviewing the same individuals with different methods. In Colombia, CTR finds that, contingent on  geographic location and cultural context, different communities---composed of members of the same two ethnic groups---differ in their expression of parochial altruism. He also finds that slightly different experimental framing---e.g., use of a giving game versus an exploitation game---can lead to different conclusions about inter-group relationships. In Bolivia, ACP finds that resource competition and past experiences with out-group members affect expression of parochial preferences, but also that careful research design is needed to disentangle preferences for long-distance relationships from preferences for between-group relationships. Further, she finds that that ``cheap talk'' does not necessarily reflect behavior in the real world. Our case studies serve not only to highlight how carefully we should test our theories---that is, how carefully we should design our methods to test our theories---but also to highlight the fact that parochial altruism is not universally present in human populations (e.g., \citep{brewer1976ethnocentrism}), even if it is universally possible.

\section{Parochial altruism and inter-group relations in two rural Colombian communities}
CTR began studying inter-group relationships between  Afrocolombian, Ember\'a, and  Mestizo communities in Colombia in 2013 \citep{ross2015frequency}. In the present study, CTR draws on demographic, social network, and experimental economic data collected in 2016--2018 to evaluate the extent of intra- and inter-ethnic social ties. Dyadic data were first collected on self-reported friendships and resource transfers. In a second wave of data collection, dyadic ties in experimental giving, exploitation, and costly punishment were assessed using the methods of \citet{gervais2017rich}. Each wave of data collection was repeated in two different communities with both Afrocolombian and Ember\'a members: a coastal community deep inside the territory where Afrocolombians constitute the strong demographic majority, and an inland community \citep[previously studied by][]{Cay73} on the boundary between a predominantly Afrocolombian population and a predominantly Ember\'a population living inside  a protected \emph{resguardo}---a kind of  reservation area with special legal standing \citep{MORAVERA2016}.



\subsection{Historical context of inter-ethnic relationships}
The contemporary ethnic make-up of Colombia is heavily influenced by historical forces related to colonization and the slave trade \citep{Can00, wade2002introduction, castillo2009discourse}.  During the late 1500s through the 1800s, Spanish colonizers transported hundreds of thousands of enslaved Africans to Colombia in order to supplement the labor being performed by the (rapidly declining)  enslaved Indigenous populations \citep{slavenote, gilbertomurillo2001el}. 
 These enslaved individuals labored primarily in gold and emerald mines, sugar cane plantations, cattle ranches, and haciendas---most notably on the Pacific coast in the states of Choc\'o and Cauca, which today remain areas with a strong demographic prominence of Afrocolombians \citep{gilbertomurillo2001el, wade2002introduction}. 

After the cessation of slavery in Colombia, the relationship between the descendants of enslaved Africans and the Indigenous peoples  of the Pacific coast could be generally characterized as one of tolerance---and sometimes even one of explicit cooperation and inter-dependence through institutions like \textit{compadrazgo}, or godparenthood \citep{Cay73}. To this day, at a national level, there remains an overarching sense of solidarity between these groups, as they have jointly fought for greater representation, visibility, and institutional support inside of Colombia \citep{castillo2009discourse, iglesiasvoces}. In practice, however, the nature of this inter-ethnic relationship appears sensitive to local contexts, especially in recent times as competing claims over resource access, land titling, and general use of ancestral lands have led to disagreements between some Afrocolombian and Indigenous groups \citep{ng2000titling, davis2002indigenous, garcia2009diversos, velasco2011contested}. 
 
At a finer-scale, inter-ethnic relations can also be quite variable within communities. In fact, ethnographic accounts from \citet{Cay73}---writing almost 50 years ago about inter-ethnic relations at the inland community, described below in section \ref{pops}---indicate both that Afrocolombians and Ember\'a have long lived in a \textit{simbiosis cultural} (with Afrocolombians, for example, commonly giving food and lodging to Ember\'a, perhaps in exchange for traditional ecological knowledge) and that it was also common to hear derogatory cross-group stereotypes voiced by members of both groups. He even notes cases of mistrust boiling over into inter-ethnic homicide. Ethnographic observations at the present time remain remarkably consistent with those of \citet{Cay73}; inter-ethnic food sharing and lodging are still daily occurrences among some members of the community, though many Afrocolombians reject Ember\'a requests for food sharing, and derogatory cross-group stereotypes are still voiced. These ethnographic accounts of the nature of inter-group relations raise the question as to how one can formally quantify and assess the overall extent of inter-group cooperation and/or animus. %Cody: I still found it surprising to suddenly see that it's common to reject demand sharing, per the answer to Q3 in the Results section, after the previous two mentions of Cayon emphasize that some share and some hold derogatory views -- not that most reject requests for sharing. I tried to at least foreshadow this in the phrase added above, but feel free to modify as appropriate.

\subsubsection{Collaborating communities}\label{pops}
In both the coastal and inland communities studied here, a large proportion of residents---Afrocolombian and Ember\'a---are considered internally displaced persons within Colombia \citep{oyola2015religion, escobar2003displacement}, having resettled after being forced from their natal communities due to violence from guerilla and paramilitary groups, mostly in the 1990s--2000s \citep{ibanez2009forced}. The coastal community is located in the Pacific region of western Colombia and relies on a mixture of artisanal fishing and local wage labor.  The inland community is located in the rainforest of western Colombia and relies on a mixture of horticulture and local wage labor.

The socio-economic situation of the \emph{in-sample} Ember\'a is similar in both communities: they are a demographically smaller group that resides on more marginal land than the \emph{in-sample} Afrocolombians, and they have comparably less access to resources like electricity, clean water, and sanitation services.  However,  \emph{out-of-sample}, at the larger administrative-district scale, the inland Ember\'a population is---in contrast to the coastal Ember\'a population---comparably well-off.  The inland Ember\'a community has a large population size, access to markets for selling hand-crafted artisanal jewelry,  and organizational connections, and most individuals (though not those in sample) reside on a resguardo.  The status of the Ember\'a at the district level has led many in-sample Afrocolombians to think that the Ember\'a are generally well-off compared to Afrocolombians, even though this is not necessarily true of the in-sample Ember\'a. The perception that the Ember\'a are well-off is virtually absent from the coastal community, where the Ember\'a population is, and is  perceived to be, living under tougher  socio-economic circumstances---on the border of a landfill, after a series of displacements.

At the district level, the coastal study area is predominately Afrocolombian (0.84 Afrocolombian, 0.09 Ember\'a, and 0.07 Mestizo; yielding a ratio of about 9.5 Afrocolombians per Ember\'a), while the inland study area rests along a three-way ethnic boundary with a less discrepant distribution of ethnic groups (0.14 Afrocolombian, 0.34 Ember\'a, and 0.52 Mestizo; yielding a ratio of about 2.5 Ember\'a per Afrocolombian) \citep{DAN051}. The demographic composition of each sample, however, is somewhat different from that of the larger district. In both study communities, the sample is composed of one comparatively large Afrocolombian sub-community ($n=88$ adults coastal, $n=130$ adults inland) and one comparatively small Ember\'a sub-community  ($n=28$ adults coastal, $n=21$ adults inland). In both coastal and inland communities, Afrocolombians have higher material wealth (average household-level wealth is about 3.7 times higher for Afrocolombians relative to Ember\'a in both communities), higher incomes (self-reported monthly income is about 1.8 and 3.9 times higher for Afrocolombians relative to Ember\'a in the coastal and inland communities, respectively), and stronger political influence at the local level. %Cody: for consistency: would "district level" be equivalent to "local level"?

Although ethnic identity is perhaps the most salient group identity in these two communities, religious affiliation can also create group boundaries that  structure cooperative or cost-generating behavior  \citep[e.g.,][]{lang2019moralizing, purzycki2016moralistic, hruschka2014impartial}. For example, rich qualitative accounts \citep[e.g.,][]{oyola2015religion} argue that beginning in the 1960s, an the Catholic church began to focus on liberation theology in predominantly Afrocolombian areas. This focus led religious groups to play a central role in: (i) unifying black social organizations,  (ii) strengthening social bonds within Afrocolombian communities, (iii) fighting for legal recognition of Afrocolombian territory as collective property, and (iv) supporting local communities in resisting forced displacement---objectives that emphasized within-group cooperation and solidarity in the context of conflict with outside actors (that is, parochial altruism) \citep{oyola2017local}.

Nominally, both the coastal and inland communities are composed almost entirely of believers in God; in CTR's sample, 51\% and 68\% of respondents identify as Catholics, 25\% and 24\% identify with no religion, and the rest identify with some other religious group, such as Christians, Evangelicals, Pentecostals, or Seventh Day Adventists. Due to sample size, rather than considering whether two individuals are of the same religious affiliation (e.g., both Catholic), we explore whether individuals exhibit parochial attitudes by favoring others who are simply of the same religious orientation (i.e., religious, or not religious) as themselves.


\subsection{Research goals}
To address the open theoretical and methodological questions outlined in the introduction, the analysis here focuses on five main questions:\\
\indent
\begin{enumerate*}[label={Q(\arabic*)},font={\color{blue!50!black}\bfseries}]
\item \label{q1} \emph{To what extent do ethnic and religious parochialism structure social relationships, altruistic giving, exploitation, and costly reduction?} While both communities are characterized by differences in ethnic group membership and religious affiliation, the ethnographic account discussed above \citep{Cay73} suggests that, at least with respect to ethnicity, such boundaries may not actually limit cooperative giving, though there is notable evidence of inter-group conflict and  negative inter-group stereotypes. To resolve this conflicting account, we use economic game, self-report, and ethnographic data to study the extent to which ethnic---and even religious---boundaries serve as containers for cooperation and cost generation in these communities.
\\
\indent

\item \label{q3} \emph{How do self-report data about food and money transfers compare to data collected with experimental economic games which each have different framings?} As discussed in section \ref{inthemethod}, subtle differences in the way an economic game is framed---for example, whether it implicitly pits in-group against out-group members---can alter participants' behavior. Using three different network-structured economic  games \citep{gervais2017rich}, each with a different framing, can participants' self-reports about their real-world giving and receiving relationships be recapitulated \citep[e.g.,][]{gurven2008collective}?\\
\indent
% Cody: here you referred to the real world data as 'altruistic giving' but that's not technically correct, as it's only altruistic if it's a one-way flow with no reciprocity; there are some bidirectional ties in your data. I changed it to sharing.


\item \label{q2} \emph{How responsive is  parochialism to varying cultural contexts---especially those owing to the relative wealth and population size of interacting groups?} If parochial altruism is indeed flexible, the literature suggests that with-group cooperation and between-group competition should be more pronounced in contexts of heightened resource competition \citep{bellmoya} and/or when the  relative population size of interacting groups is more balanced \citep{advani2015melting}. Likewise, formal models of social interactions in ethnically-structured populations \citep{mcelreath2003shared, bunce2018sustainability} predict that if there are differences in population numbers between two ethnic groups, the salience of ethnic markers---and the corresponding preference to interact within one's own ethnic group---will be stronger at an ethnic boundary (i.e., the inland site in this study) than at a site deeper inside the territory of a given ethnic group (i.e., the coastal site in this study). %Cody: there seem to be three different ideas floating around here: presence of a boundary, density, and (per the question itself) population size, each of which is independent of the other. I decided to rework this sentence for simplicity, but have preserved it here in case you disagree. (Also, note that my read of Bunce & McElreath is that it's population numbers not density.)
%Likewise, formal models of social interactions (as coordination games) in ethnically-structured populations \citep{mcelreath2003shared, bunce2018sustainability} predict that if there is clinal geographic variation in relative population density, then the salience of ethnic markers---and the corresponding preference to interact within one's own ethnic group---will be stronger at an ethnic boundary---i.e., the inland site in this study---than at a site deeper inside the territory of a given ethnic group---i.e., the coastal site in this study.
 Comparing the coastal and inland communities, are there differences in the expression of parochial altruism that can be attributed to the differing cultural context of inter-ethnic interactions? If so, what about the cultural context best explains such differences?\\
 \indent
 
\item \label{q4} \emph{Are preferences for in-group cooperation decoupled from preferences for out-group exploitation and cost imposition?} As discussed in section \ref{onepointone}, existing data suggest that preferences for in-group cooperation are common, but  not necessarily associated with animosity towards out-group members. By using social network data that separate the generation of benefits and costs, the possible independence of preferences for in-group favoritism and out-group cost imposition can be assessed. For example, while the RICH allocation game implicitly pits in-group against out-group members by providing only a few coins to allocate across many community members, the RICH taking/exploitation game provides enough coins such that all community members can receive one. %Cody: the game has three names. To keep things simple, maybe ditch one (exploitation). It gets really confusing by the results, where explotation disappears and is replaced by leaving.
Considering economic game, self-report, and ethnographic data, is it the case that individuals express both preferences to direct cooperation towards in-group members and to direct costs towards out-group members, or are such preferences actually decoupled in the rural Colombian context?
\\
\indent

\item \label{q5} \emph{To what extent can apparent parochial altruism be explained by focal, alter, and dyadic covariates?} %You mention "alter" but don't return to this in this paragraph. Are alter characteristics also of interest? Also, note that the term "alter" hasn't been defined yet.
Across real-world populations, inter-personal relationships are influenced by many variables---e.g., kinship and relative socio-economic status---that have the potential to covary with markers of ethnicity. For example, if individuals are more likely to give to kin than to non-kin, and kin are of the same ethnicity, then giving may appear to be directed toward co-ethnics even if ethnic identity \emph{per se} affords no special consideration in resource transfers. As such, in the Colombian context, are estimates of parochialism in network ties from self-report and game play data robust to controls for individual-level (focal) characteristics (e.g., wealth and food security), as well dyad-level characteristics (e.g., kinship and  marriage)? %Cody: I feel like you reworked this, and I like it! Good question.

\end{enumerate*}


\subsection{Methods}
CTR used community-wide censuses (in 2016--2017) to obtain social network, demographic, and socio-economic data. He used name generators which asked participants to identify X friends and X individuals to whom they transfer food or money. These data are paired with data from three Recipient Identity-Conditioned Heuristic (RICH) economic games developed in \citet{gervais2017rich}, which were run in a subsequent field season (2017--2018). RICH games involve tasks in which participants (a.k.a. deciders) have a chance to: (i) allocate money to, (ii) take money from, and (iii) at a cost to themselves, reduce the payouts to other members of their communities (a.k.a. alters/recipients). Unlike classical economic games, like the Dictator or Third-Party Punishment games, RICH games do not involve making economic decisions with respect to anonymous recipients. Instead, deciders are presented with an array of recipient photographs and thus know who they are giving money to, or taking money from, but recipients do not learn who gave to, or took money from, them. 

The three RICH games have important differences in framing. In the allocation game, participants were given a small number of 1,000 \textit{peso} coins (15 at the coastal site and 20 at the inland site), which they could keep or allocate across in-community alters---$n=116$ in the coastal community and $n=151$ in the inland community. In the taking/exploitation game, CTR placed one 500 \textit{peso} coin on the photo of each of the alters, which deciders could leave in place or take for themselves by engaging in exploitative behavior \citep{bhui2019exploitation}; the maximum a decider could leave was one coin for \textit{each} alter. Finally, in the costly reduction game, CTR gave the decider a small number of 1,000 \textit{peso} coins (10 at the coastal site and 15 at the inland site) which they could keep or use to reduce any alter's income by 4,000 \textit{pesos} for each 1,000 \textit{peso} coin spent. Costly reduction is generally indicative % "interpreted as" instead of "indicative of"?
of inter-personal animosity \citep{gervais2017rich}, but the underlying motivation for such behavior can be variable. Some individuals might use costly reduction to engage in  norm enforcement \citep{Fehr2002}---e.g., punishing excessively wealthy in-group members \citep{gervais2017rich, Pisor2020} or in-group members with bad or selfish reputations \citep{bhui2019exploitation}, while others might use costly reduction simply to express animus towards individuals of ethnic or religious out-groups. % Cody: yep, this makes sense to me! I'm on board.

In total, CTR modeled five outcome networks---from the self-report data, friendship ties and resource transfer ties, and from the RICH games, decisions in the allocation, taking/exploitation, and costly reduction games---using a Bayesian social network analysis based on the Social Relations Model \citep[SRM,][]{kenny1984social,  koster2019statistical} and implemented in Stan \citep{Rstan2017xx}. Predictors are inspired by \ref{q1}-\ref{q5} and include characteristics of the focal respondent and the set of alters (including their ethnicity, religious orientation, and wealth), as well as characteristics of the dyad (including whether individuals are married to each other or genetically related). To examine differences in parochial altruism between the coastal and inland communities, effects unique to each site are estimated separately. Because the game poses an opportunity to be generous by leaving coins for others, the taking/exploitation game is reverse coded as a leaving game. For further details on data collection protocols, game design, and statistical analyses, see supplementary appendix sections X--X.

\subsection{Results}
\subsubsection{Quantitative findings}
The results of model fitting are visualized in Figure \ref{colombianres}. Within each column, blocks show the standardized effects of focal, alter, and dyadic characteristics on the likeliness of having a tie with, allocating money to, taking money from, or reducing the payout of an alter. Parameter estimates are blue for the coastal community  and orange for the inland community. Dark orange and dark blue bars give the same estimates while accounting for the full set of controls (that is, all of the predictors listed in the figure) in a multivariate model. % Cody: the difference between the oranges is kind of subtle. Might be easier to see with a different color.
Of principal interest to our research questions  are the effects in the \emph{Parochial} block (row 4). In this block, light orange and light blue bars plot estimates from models without control variables---i.e., those that included either the ethnicity or religiousness of self and alter (variables reported in block 1), with no other covariates. The effects of control variables are similar to those described in \citet{Pisor2020} and are largely similar between communities. A more detailed discussion of Figure \ref{colombianres} is presented in the supplementary appendix. Here, we focus  only on addressing the estimates relevant to questions \ref{q1}-\ref{q5}:

\ref{q1} \emph{To what extent does ethnic and religious parochialism structure social relationships, altruistic giving, exploitation, and costly reduction?} While ethnic identity and religiosity both structure social relationships and game play, the effect of ethnicity is more pronounced, especially in the inland community. From the perspective of Ember\'a individuals in both communities, ties---be they friendships, transfers of food or money, or transfers of coins in the RICH allocation and taking games---are more likely to be directed towards other Ember\'a than toward Afrocolombians. Afrocolombians at the inland community also preferentially form friendships with and give to (both inside and outside of the game context) other Afrocolombians; however, this effect only partially holds in the coastal community, where food and money transfers (column 2) as well as transfers in the taking/leaving game (column 4) show no evidence of parochial preferences.

 Religious individuals in the coastal community are more likely to make transfers of food and money to other religious individuals (column 2), are more likely to allocate coins to other religious individuals in the allocation game (column 3), and are less likely to punish other religious individuals in the costly reduction game (column 5). These effects, however, do not replicate in the inland community, where religious individuals were more likely to transfer food and money to \textit{non-religious} individuals. 
 
 In short, ethnic group membership clearly structures social relationships and game play at both sites, especially for the Ember\'a, while religiosity has less consistent and pronounced effects. Lastly, there is no clear ethnic pattern of costly reduction directed at either in-group or out-group members in either community.


\ref{q3} \emph{How do self-report data about food and money transfers compare to data collected with experimental economic games which each have different framings?} To answer this question, it is useful to first compare the effects described in the self-reported food and money transfers model (column 2) to those described in the allocation game model (column 3), as both effects correspond to making resource transfers in a context of economic constraint---in the real world, an inability to give food or money to everyone, and in the game context, too few coins (either 15 or 20, depending on site) to give to all alters. Many predictor variables, especially the dyadic ones (row 3), are consistent between these outcomes. The results also suggest fairly similar patterns of parochialism (row 4), with the qualification that there is weaker evidence of parochialism in food/money transfers than in allocation game decisions among Afrocolombians in the coastal community. Turning our attention to parochialism in the leaving game (column 4), where deciders could leave coins for \textit{all} alters if they so chose, we again see comparability between predictors of self-reported transfers and predictors of coins left for other, supporting the external validity of the RICH games.

\ref{q2} \emph{How responsive is parochialism to varying cultural contexts---especially those owing to the relative wealth and population size of interacting groups?} When comparing the parochial altruism demonstrated by participants from the coastal and inland communities (row 4), the effects in the leaving game (column 4) and self-reported food/money transfer models (column 2) stand out. In the leaving game, where coins \textit{taken} benefit the decider at the expense of an alter, both Afrocolombians and Ember\'a at the inland community are more likely to take coins from out-group members than in-group-members; however, on the coast, model estimates suggest that Afrocolombians are either just as likely to leave coins for Ember\'a  as for Afrocolombians (model with controls) or are \textit{more} likely to leave coins for Ember\'a than for Afrocolombians (model without controls).  Afrocolombians in the inland community show parochialism in small food/money transfers---a fact that may reflect a common \citep[although not universal,][]{Cay73} rejection of inter-ethnic demand sharing requests---while Afrocolombians in the coastal community show no such parochial preference and commonly engage in inter-ethnic giving. %Cody: the clarification of transfers being "small" was news. Can this be added to the methods if it was specifically a question about small transfers?
 We discuss further qualitative evidence concerning these key findings and provide more details about the differences in cultural context underpinning these between-community differences in parochialism in section \ref{qual}. 

\ref{q4} \emph{Are preferences for in-group cooperation decoupled from preferences for out-group exploitation and cost imposition?} Social-network data based on giving---e.g., food/money transfers (column 2) and allocation game play (column 3)---or  on-going relationships---e.g., friendship ties (column 1)---can provide insight into in-group favoritism. To gain insight into negative ties, it is useful to study other kinds of behavior, like exploitation---e.g., the leaving game (column 4)---or direct cost imposition---e.g., the costly reduction game (column 5). Data on food/money transfers, allocation game play, and friendship ties generally suggest in-group favoritism with respect to ethnic group members, although not (necessarily) with respect to alters who are similarly religious. %Cody: can you unpack the religious results with a brief e.g.?
In other words, same-ethnic-group favoritism in time and resource allocation appears quite robust in this data set. Additionally, as discussed above,  inland Afrocolombians and Ember\'a at both sites are also more willing to generate costs for out-group than in-group members in the leaving game, consistent with the predictions of models of parochial altruism.  However, we see no evidence of elevated out-group exploitation among coastal Afrocolombians. We also see no evidence of out-group bias in costly reduction in either ethnic group, in either community. As such, parochialism in out-group exploitation and especially out-group cost imposition seems to  be decoupled from parochialism in within-group cooperation in these communities.

\ref{q5} \emph{To what extent can apparent parochial altruism be explained by focal, alter, and dyadic covariates?}
We find that estimates of parochialism (row 4) are surprisingly robust to the inclusion of controls for material wealth, food security, marriage ties, and genetic relatedness that could otherwise generate ``epiphenomenal'' parochialism, especially in contexts, like that of the RICH allocation game, where the set of resources that can be distributed is much smaller than the set of possible recipients. In one case to the contrary, however, the apparent  ``anti-parochialism'' in leaving coins---i.e., the preferential leaving for Ember\'a alters by Afrocolombians in the coastal community---is attenuated when accounting for control variables. As these control variables include the material wealth and food insecurity of the alter, the reduction in the effect size of ``anti-parochialism'' upon inclusion of controls might indicate a mediating role of economic need in driving transfers from Afrocolombians toward Ember\'a at the coastal site.

%Cody: Maybe move up the figure? It comes so far after the results in the text that flipping back and forth is kind of a chore.
 \begin{figure*}[t]
 \centering
\includegraphics[width=7in]{All_Games-Standardized_SRM} % this command will be ignored
\caption{{\footnotesize Multinomial regression results (with standardized coefficients) from the Social Relations Model. Points and line-ranges show the standardized effects of predictor variables on outcomes (as medians and 90\% credible regions of the posterior distributions). Each column indicates an independently modeled outcome variable: from left to right, i) friendship/socializing ties over the 30 days prior to the initial survey, ii) resource transfers (including food and money) over the 30 days prior to the initial survey, iii) coin allocations in the allocation game, iv) coin deductions in the taking game (coded so that positive parameter estimates reflect \textit{leaving} coins), and v) coins paid to reduce an alter in the costly reduction game. For each of these outcomes in each community, CTR fit two models: both included the predictors directly related to parochial altruism (e.g., as in row 4), but the first (NC; \textit{No Controls}) excluded control variables (that is, the predictors in all other rows, except being religious and being Ember\'a (row 1)) and the second included such controls. The first three rows report control variables associated with the focal (decider), the alter, and the dyad; the fifth row provides estimates of reciprocity between the focal and alter (e.g., whether they share a bidirectional friendship tie).}
} \label{colombianres}
\end{figure*}

\subsubsection{Qualitative accounts}\label{qual}
In post-game interviews, in both communities, the most common explanation for game play behavior was the heuristic: \textit{take from those who are better off and can afford it, and leave for those who are worse off and  need the money more. } In this light, the across-the-board parochialism of Ember\'a participants in both communities is explainable by the fact that, compared to local Afrocolombians, the Ember\'a live on more marginal lands, under more precarious economic circumstances, and in smaller, closer-knit groups where need and well-being are known. As we will see in section \ref{boliviagame}, just like the Tsimane' in Bolivia, Ember\'a give to, and leave for, those they see as most in need: other Ember\'a.

When asked in post-game interviews to explain their rationales for taking from whom they did, it was common for coastal Ember\'a respondents to emphasize taking from ``those who already have money to live on'' or ``those who have jobs,'' and leaving for ``people  in similar or worse situations to [themselves]'' and ``[their] neighbors who are also poor.'' Some coastal Afrocolombians also specifically mentioned out-group ethnicity as a motivation for \textit{not} taking coins: ``[I left coins for] the indigenous, the sick, and people of old age'', grouping Ember\'a residents into the class of people deserving of special consideration. Afrocolombians and Ember\'a at the coastal site both agree on whose relative need is greater; accordingly, Afrocolombians did not show evidence of parochialism in real-world food/money transfers or in experimental exploitation decisions in the RICH taking game. Qualitative responses in the coastal community focused on objective need and carried little emotional valence.


Similarly, in the inland community, Ember\'a participants agreed that Ember\'a alters were more in need than Afrocolombians and biased giving towards other Ember\'a accordingly. In stark contrast to the coastal community, however, it was very common in post-game interviews for Ember\'a respondents to describe taking from alters (normally Afrocolombians) specifically because those alters had not cooperated in the past, and there was clearly more social friction and negative emotional valence than in the coastal community. Ember\'a respondents would state that they took coins: ``because these  are bad people  who don't cooperate'', ``because those people don't cooperate with you when you ask for help,'' or ``because they are bad people. You are hungry and ask for a favor and they do nothing.'' Though not recorded as explicitly in Afrocolombian's post-game interviews,  Afrocolombians in the inland community often imputed that Ember\'a alters from their community were as well off as the Ember\'a living in the nearby resguardo, causing them to engage in fewer inter-ethnic need-based or demand transfers (that is, giving food or money when asked), as is clear in both the experimental leaving game and real-world food/money transfer data.

Although some of the content of the post-game interviews was similar at both coastal and inland sites---e.g., Ember\'a respondents would frequently describe taking from Afrocolombians who are comparably better off---the general tone was very different. This difference seems to arise from Ember\'a and Afrocolombian respondents at the coastal site having a shared perception of the correlation between need and ethnic group, while Ember\'a and Afrocolombian respondents at the inland site have different perceptions of who is most in need---and thus whether there is any obligation for Afrocolombians to provide food or money if requested by Ember\'a (or, similarly, whether there is any obligation to refrain from exploitation in the experimental taking game).  

The findings presented here, both quantitative and qualitative, contrast in some ways with the \textit{simbiosis cultural} reported by \citet{Cay73}. By integrating community-wide self report and experimental data, along with qualitative debriefing interviews and standard ethnography, we have been able to build
a more representative understanding of inter-group relationships. At a large-scale, the characterization of inter-group relationships as a \textit{simbiosis cultural} remains valid, as direct hostilities between groups are quite rare, and in both locations inter-group cooperation occurs between specific individuals on an almost daily basis. However, focusing only on a handful of easily observable cooperative relationships would obscure the larger picture, that on average, relationships in the inland site are rather parochial.
  

\subsection{Discussion}
\subsubsection{Mixed evidence for  parochialism}

Consistent with our review of the parochial altruism literature (section \ref{onepointone}), our findings here demonstrate mixed evidence for ethnic and religious parochialism---across communities, groups, and data elicitation methods. %But, by tying together the partial insights offered by observational, experimental, and ethnographic data, we can gain a fuller understanding of inter-group relationships and the norms that govern cooperation and exclusion.

Friendship ties based on time spent socializing  suggest a high degree of social assortment on the basis of group identity; this parallels similar findings at other sites \cite[e.g.,][]{power2017social, baerveldt2007ethnic}. Despite Afrocolombians and Ember\'a living in close proximity to each other in both communities, socializing is primarily confined to within-ethnicity interactions. These data correspond to historic accounts of a paucity of inter-ethnic marriages despite a long history of social contact \citep{Cay73}, and genetic evidence that shows a high degree of population substructure in the Pacific region of Colombia, in contrast to the Caribbean region of Colombia where genetic admixture is high \citep{ossa2016outlining}.

Consistent with the social network data, when a small windfall of money was provided to respondents in the RICH games, this money was allocated primarily to same-ethnicity alters---an effect that was robust to controls for other measures of social proximity, including marriage and kinship. %Under  experimental resource constraints, where the number of coins to allocate  was much smaller than the number of possible targets, parochial preferences were clearly distinguished.
Based on such findings, one might conclude that individuals have a general predisposition to direct time and aid to coethnics.  However, the observed cross-site variation in parochialism in food/money transfers and exploitation ties highlights the flexible nature of the in-group/out-group distinction and illustrates how our choice of method might influence the weight of evidence for parochial altruism. Moreover, data on costly reduction suggest an absence of direct inter-group animus in both sites. In sum, the data suggest that expression of parochialism is context dependent, sensitive to the method of measurement, and that positive ties to same-ethnicity alters can be decoupled from negative ties to other-ethnicity alters.

\subsubsection{Relative population size, resource competition, and the cultural context of interactions}
\label{contextcc}
The coastal community is characterized by Afrocolombians having higher population size, more stable land tenure,  stronger local political institutions, and greater control of the means of production (i.e., fishing boats, refrigeration, market contacts). The inland community, on the ethnic boundary, in contrast, is characterized by both Afrocolombians and Ember\'a both having substantial population sizes, stable land tenure, strong local political institutions, and more comparable bargaining power. 

Although between-group resource competition is thought to be an important driver of parochialism  \citep{bellmoya} and there is greater scope for such competition at the inland site where population size and institutional power are more balanced, this explanation is not particularly relevant for these two sites because between-group resource competition is not a central feature of the cultural context. This being said, resource competition \textit{has} been cited for the breakdown of inter-ethnic cooperation between these same two ethnic groups in other regions of Colombia \citep[e.g.,][]{ng2000titling, davis2002indigenous, garcia2009diversos, velasco2011contested}. As such, it is possible that in other rural Colombian contexts with more intense inter-group competition---i.e., where there is direct legal conflict over land or resources---greater parochialism in positive and negative ties would arise.

Likewise, our contrast of inter-group relationships at the coastal and inland sites is concordant with the predictions of some formal models of inter-ethnic coordination games \citep[e.g.,][]{mcelreath2003shared, advani2015melting}, which argue that ethnically-based social assortment is most likely to emerge when the population size of both ethnic groups is sufficiently large, such that within-group interactions occur frequently enough in both groups to maintain sub-populations with distinct behavioral norms \citep{bunce2017interethnic, bunce2018sustainability}---a circumstance that is predicted to arise in communities on the geographic boundaries between groups \citep{mcelreath2003shared}. Again, however, there is no ethnographic indication that cross-site variation in parochialism is driven by such dynamics. Instead, the role of relative population size on expression of parochial preferences seems to be linked to the widely shared understanding that transfers should be based on relative need.




\subsubsection{Need-based transfers and inter-group relations}\label{discneed}

In both communities, formal statistical analysis and qualitative post-experiment interviews identified a key norm governing transfers:
 \textit{take from those who are better off and can afford it, and leave for those who are worse off and need the money more}---a classic, need-based heuristic found across a variety of cultural groups
\citep[e.g., ][]{peterson1993demand, hooper2015inclusive, aktipis2016cooperation, hao2015need, gervais2017rich, cronk2019managing}. This need-based norm appears more salient to respondents than a group-identity based norm. %Similarly, comparing individuals who have more resources to those who have less resources (or, importantly, who are perceived to have less resources), the haves stand to lose less by giving food or money than the have-nots stand to gain by receiving them \citep{winterhalder1996marginal,winterhalder1997gifts}. This is the basis for what some call demand sharing or tolerated theft.

%It does not appear that there is any disagreement between respondents in either community as to what norm should be used to govern resource transfers during dyadic interactions: the above-mentioned need-based heuristic is widely agreed upon. 


Inter-ethnic, need-based transfers, like those described at the coastal site, may be explainable by models of tolerated theft. Differences in the marginal valuation of small amounts of food/money between individuals with different resource holdings can be large \citep{jones1984selfish, winterhalder1996marginal}. Likewise, the marginal fitness benefit to a resource-poor individual in a comparatively impoverished group receiving a small food/money transfer may be much larger than than the fitness cost of making such a transfer to a resource-rich member of a comparatively well-off group  \citep{winterhalder1997gifts}. In the context of inequality, imbalance in marginal valuation has the potential to lead to conflict, as the resource-poor individual may be willing to escalate their demands in order to secure an essential resource; this dynamic can lead to need-based transfers when a well-off giver shares a resource whose benefit (to the receiver) exceeds the cost (to the giver) of defending that resource.

 If ethnicity and need are perceived to structurally covary, then members of a relatively well-off sub-population may use ethnicity as a marker to direct need-based transfers---attenuating overt expression of parochialism. In the coastal community, Ember\'a make up a small proportion of the population and there is large between-group, but little within-group variation in wealth or influence; ethnicity thus covaries strongly with perceived need and conveys key information about individuals in the minority group. Coastal Afrocolombians and Ember\'a both recognize that the obligation to help to most needy entails that resources should flow towards Ember\'a.  In the inland community, however, district-level population sizes are more balanced, and within-group variation in wealth and influence (e.g., comparing Ember\'a in the community to Ember\'a living on the resguardo) are higher. Here, ethnicity does not covary with perceived need and thus fails serve as an indicator that can be used to guide transfers. As such, inland Afrocolombians and Ember\'a are more likely to claim that their obligation to help to most needy entails that resources should flow towards members of their own group. Relative socio-economic status thus plays a key role in explaining variation in cooperative behavior \citep[see also, ][]{silva2014cooperation}.

%However,  The greater population size of Ember\'a residents at the inland site affords greater anonymity to Ember\'a individuals there. This social distance, coupled with higher within-group variance in wealth and status, makes it more difficult for inland Afrocolombians to infer relative need on the basis of ethnicity, and thus reduces inter-group demand transfers---recall that inland Ember\'a report frustration that a need-based heuristic is not being employed when they ask for assistance---and elevates social tensions across the ethnic boundary relative to what is observed on the coast.

%In the inland community, it seems probable that social relationships are more highly structured by ethnicity than on the coast---perhaps due to the mechanism highlighted by \citet{mcelreath2003shared}. Indeed, it is more common to observe inter-ethnic play among both children and adults (e.g., in community soccer games) in the coastal community.

%\subsubsection{Cultural context, the salience of ethnicity, and inter-group relations}

%Following models of social interactions as coordination games in ethnically marked populations---that is, populations in which there are markers that can be perceived (e.g., seen, like special dress, or heard, like a dialect) that indicate group membership \citep{mcelreath2003shared}, we would expect the salience of ethnicity and ethnic markers to be stronger at an ethnic boundary (i.e., the inland community) than at a site deeper inside the Afrocolombian territory (i.e., the coastal community). At a site where one group constitutes a strong demographic majority, frequency-dependent adoption of norms/behavior will tend to drive the norms/behavior of the majority population to fixation, even in the minority ethnic group, unless special conditions are met \citep{bunce2017interethnic, bunce2018sustainability}. This dynamic generally decreases the covariance between norm and marker, and thus the salience of ethnicity to social interactions in such cultural contexts.

\subsubsection{Religious parochialism}\label{relig}

%Theoretical accounts of the role that religion is expected to play on cooperation with out-group members are still incomplete and largely verbal. On the one-hand, \citet{norenzayan2016cultural}, echoing \citet{choi2007coevolution}, suggest that religious parochialism should be favored by group-level selection (relative to both indiscriminate prosociality and indiscriminate selfishness), because such a norm leads parochially altruistic groups to out-compete other groups, especially in contexts of direct, group-level conflict over resources \citep{lang2019moralizing}. However, the same authors also predict that in the absence of direct conflict between groups, a relaxation of parochialism---or even cooperation directed to the out-group---may be favored, as it could lead to more religious converts, an expanded cooperative circle, and a wider uptake of cultural beliefs (like supernatural punishment) that stabilize cooperative behavior with anonymous alters (e.g., as found in market-integrated societies with a large fraction of individuals believing in deities that can supernaturally monitor and punish defectors) \citep{purzycki2018evolution, henrich2010markets, lang2019moralizing}.

Parochialism on the basis of religious group membership also appears to be flexible and responsive to cues of inter-group conflict---e.g., over land or resources. For example, reviewing the literature, \citet{lang2019moralizing} conclude that ``some studies showed that participants affiliated with religions emphasizing universal morality embrace the extension of cooperation behavior to out-group members \citep{preston2013different, ginges2016thinking, clingingsmith2009estimating, mccullough2016christian}, while other studies indicated that religious participants reveal hostility toward religious out-groups \citep{bushman2007god, shaver2018boundaries}'' [p. 2]. 

In the coastal community, both food/money transfers and game allocations were more likely to be directed, and costly reductions less likely to be directed, to other religious individuals. These effects did not replicate in the inland community, where food/money transfers were actually more likely to flow from religious individuals to non-religious individuals. It is possible that deeper within the Afrocolombian territory (i.e., where the coastal community is located), a focus on within-group solidarity coupled with a cultural context of resistance to hostile actors \citep{oyola2017local} %Cody: unpack briefly to connect to your overview of the community. You mean standing up to the government for recognition or resistance to displacement or guerillas or...?
 served a key role in fortifying relationships among the religious in-group. However, (i) sample size was insufficient to assess whether in-group favoritism was specifically directed toward members of the same church (e.g., the Catholic church), and (ii) more detailed ethnographic investigation of the churches of interest, along with targeted post-experimental interviews with religious respondents, would be needed to validate such explanations.

Institutions can encourage the generation of between-group benefits and the reduction of between-group costs (section \ref{onepointone}), and religious institutions are no exception \citep{lang2019moralizing}. As individuals attend Catholic and Evangelical churches at roughly the same rates in both communities (section \ref{pops}), we cannot attribute the difference in ethnic parochialism between the coastal and inland Afrocolombians to differences in the prevalence of anti-parochial norms seeded by churches. That said, we cannot rule out the possibility that one or several religious leaders are spreading more tolerant views in the coastal community.


\section{Inter-group relationships vs long-distance relationships among lowland Bolivian horticulturalists}
\subsection{Assuming parochial altruism}
ACP began studying inter-group relationships in Bolivia in 2010, with a focus on why and when individuals form cooperative relationships with people outside of their communities.   She focused on Bolivia for two reasons. First, after three decades of large-scale movements pushing for Indigenous rights,  in 2009 the country became The Plurinational State of Bolivia. The federal government recognizes the sovereignty of 36 different  \textit{pueblos ind\'igenas}---Indigenous groups whose members are \textit{originarios}, living on their traditional lands, and whose members share cultural institutions (explicitly recognized by the government as the pueblo's \textit{usos y costumbres}), making ``pueblo ind\'igena'' akin to the definition of ``ethnicity'' used in the parochial altruism literature. However, because the state preferentially allocates its limited funds to originarios, new lines of competition have been drawn between Indigenous groups. What was a superordinate identity of indigeneity in the 1980s--2000s has splintered as pueblos ind\'igenas compete with one another for resources and recognition \citep{fontana2014indigenous}.
Second, Evangelical churches of various denominations are expanding their presence in Bolivia \citep{lesley1993religious}, as elsewhere in Latin America \citep{stoll1990latin}; rural Bolivians candidly contrast Catholic and Protestant beliefs, distinguishing what ``we'' do from what ``they'' do that ``we'' would never do. 

In the midst of these changes, market participation is on the rise among slash-and-burn horticulturalists living in the lowlands as these populations rely less on subsistence production and more on cash income or trade to acquire food and other goods \citep{gurven2015does, reyes2010integration}. With increased market participation comes increased mobility, contact with middlemen, and exposure to individuals in other pueblos ind\'igenas and who live far away \citep{pisorjones2020}.

 Adopting the assumption from the parochial altruism literature that ethnic group boundaries are containers for cooperation (section \ref{onepointone}), an assumption supported by the emphasis placed on pueblos ind\'igenas by the Bolivian government, ACP set out to study why and when cooperative relationships might transcend the boundaries of pueblos ind\'igenas or, given its increased relevance, the Catholic/Evangelical divide. She hypothesized that individuals with fewer resources might be more likely to exhibit a preference for building relationships with out-group members in addition to in-group members---in other words, that their interest in resource access might act as an ``override'' holding parochial altruism in check \citep{pisor2016risk, pisor2018diversify} (section \ref{onepointone}). This hypothesis was partially supported. However, it was short-sighted.

\subsection{Collaborating communities in Bolivia}\label{pops2}
Three populations of slash-and-burn horticulturalists---the Moset\'en, the Tsimane’, and the Interculturales---are the focus of the present discussion. The Moset\'en and Tsimane’ are pueblos ind\'igenas, recognized by the Bolivian government as originarios with their own set of usos y costumbres because they live on their traditional lands. The Moset\'en and Tsimane’ have lived in the lowlands for centuries, according to archaeological and ethnohistoric data \citep{godoy2015natural, tomas2008tsimane}, and the two pueblos ind\'igenas were once a continuous, intermarrying population, as supported by genetic, cultural, and linguistic data \citep{bert2001major, godoy2015natural, gurven2007mortality, sakel2011moseten, ringhofer2010exploring}. 

Today, however, their lives are quite different. While the Moset\'en were missionized by Franciscan Catholics during the early- to mid-19th century \citep{godoy2015natural, mamani2010tsinsi, ref947717999}, the Tsimane' were missionized in the mid-20th century by Evangelical Christians \citep{tomas2008tsimane}. Efforts by missionaries resulted in access to roads and secondary education for most Moset\'en communities by the year 2000 \citep{pisor2018diversify}; in contrast,  only a minority of Tsimane' communities have access to major roads or secondary schooling \citep{ringhofer2010exploring}. 
Because of their enhanced access, the Moset\'en have more years of education, participate more in the market economy, and have higher mobility than do the Tsimane'. Today, while the Moset\'en speak fluent Spanish, the \textit{lingua franca} among the different pueblos ind\'igenas of Bolivia, and intermarry extensively with other pueblos ind\'igenas \citep{pisor2018diversify}, only 14\% of the Tsimane' speak fluent Spanish \citep{pisor2016risk} and few have intermarried with other pueblos ind\'igenas.

The Interculturales with whom ACP collaborates are another community of horticulturalists of Indigenous descent. The word \textit{intercultural} is a designation used by the Bolivian government to recognize communities of peoples from different pueblos ind\'igenas who are no longer on their traditional lands \citep{albo2007bolivia}; they are thus not considered originarios and are not eligible for special government recognition and resources. The Intercultural community discussed here is composed primarily of descendants of the Aymara and Quechua pueblos ind\'igenas. These families moved to the community as part of government relocation programs in the 1950s--60s or during booms in the logging and quinine industries  \citep{pisor2016risk, pisor2018diversify}. Upon arrival, many Interculturales learned horticulture for the first time, sometimes by copying the Moset\'en. However, they retained many Aymara-influenced institutions, especially with respect to social and political organization. Today the Interculturales are more reliant on the market economy than are the Moset\'en: because they have had reliable access to roads for 25 years longer \citep{Llojlla2011}, they began to build relationships with middlemen who buy and sell produce much earlier \citep{pisorjones2020}. On average, the Interculturales have as much education as the Moset\'en but slightly higher incomes, more market possessions, and higher mobility. By definition, they intermarry at higher rates with other pueblos ind\'igenas than do the Moset\'en.
		  

\subsection{A game with many interpretations}\label{boliviagame}

In 2014 and 2015, ACP and Michael Gurven used a non-anonymous economic game to assess whether members of the Moset\'en, Tsimane', and Interculturales were less likely to favor in-group members---individuals of the same pueblo ind\'igena or religious affiliation---if they stood to gain more from out-group members, whether in access to income or to market goods \citep{pisor2016risk, pisor2018diversify}. ACP interviewed participants from two Moset\'en communities, three Tsimane' communities, and one community of Interculturales. Participants were presented with photos of six strangers, three from their in-group and three from their out-group, and told the first name, age, and either pueblo ind\'igena or religious affiliation of each. ACP then placed three coins (each worth \$0.14 USD; total stakes: 1/3 of a day's wages) on each photo, and three coins in front of the participant; she told them that they could move any coins they wished, and that any coins left on a photo would be given to that person in the participant's name. Any coins left in the pile in front of the participant would be theirs to keep. The intention of this method was to assess participants' preferences for forming a new relationship with an in-group stranger versus an out-group stranger through an act of generosity, in a task that explicitly pitted in-group favoritism against the possibility of out-group relationships. For further details on this method, see \citet{pisor2016risk, pisor2018diversify}.

ACP and Gurven found that participants who felt that they were subjectively less well-off relative to others in their community were more likely to give money to out-group members \citep{pisor2016risk}. Though participants gave more to in-group members than to out-group members on average, indicating some preference for building relationships with in-group members, mean out-group giving was far from zero: 82\% of participants gave at least some money to out-group members \citep{pisor2018diversify}\footnote{This may have been due to priming concepts of fairness, as the initial allocation was three coins per person. That said, participants were unlikely to award 3 coins to each recipient; instead, most moved at least one coin, perhaps because they felt they had to do something \citep{list2007interpretation}. Tsimane' participants also behaved quite differently from the Moset\'en and Interculturales.}. Per our discussion of the role of religious institutions (section \ref{relig}), frequency of church attendance predicted giving more to out-group members, but only for the Interculturales \citep{pisor2016risk}. However, these patterns of out-group giving had caveats. First, Tsimane' preferences looked quite different from Moset\'en and Intercultural preferences. Tsimane' participants were far less likely to give any money to recipients of other pueblos ind\'igenas or a different religious affiliation than were the Moset\'en or Interculturales---and note that at the time of data collection, almost all Tsimane' affiliated themselves with the same Evangelical church, such that those of the same religious affiliation were also often Tsimane' \citep{pisor2018diversify}. Perhaps this was a product of the use of money in the experimental task; the Tsimane' have less wealth than the Moset\'en or Interculturales and see themselves as the ``have-nots'' relative to other pueblos ind\'igenas \citep{pisor2018diversify}. However, while participants from all three populations who felt they had fewer resources than others gave more to out-group strangers, only the Tsimane' gave more to out-group strangers if they had less invested in market items than other participants in the same population (Figure \ref{boliviamarket}). Finally, group membership was not the only basis for decision-making. In post-decision interviews, participants---especially Moset\'en and Intercultural participants---often described inferring recipient characteristics from their ages or their photos, including their relative need and whether they were a good person, in order to make decisions \citep{pisor2018diversify, Pisor2020}.

 \begin{figure}[t]
	\centering
	\includegraphics[width=3in]{Fig2FakeData}
	\caption{{\footnotesize Predicted \textit{bolivianos}, the local currency in Bolivia, given by a participant to an out-group stranger (that is, a member of another pueblo ind\'igena or religion) by the total estimated value of market items owned by a participant, normalized relative to other participants in the same population \citep{pisor2016risk, pisor2018diversify}.}} \label{boliviamarket}
\end{figure}

This contrast in generosity toward out-group members was foreshadowed by our overview of these three populations in section \ref{pops2}: the Tsimane' have fewer interactions with members of other pueblos ind\'igenas than do the Moset\'en or Interculturales. Some of this is due to their constrained mobility, given their limited access to roads and to income to purchase gasoline for river travel. Some of this is due to their lack of access to education, keeping the percentage of fluent Spanish speakers low. But some of this is also due to choice. At the time of European contact, the Tsimane' were well-known among other Indigenous groups in the region as salt traders \citep{godoy2015natural,  ref947717999}. However, their interactions with highland Bolivians (\textit{collas}) and non-Indigenous lowland Bolivians (\textit{cambas}) have been marked by misunderstandings, marginalization, and discrimination; the Tsimane' physically retreated from contact with the Spanish and cambas multiple times when these groups took advantage of them \citep{godoy2015natural, ringhofer2010exploring, tomas2008tsimane}. Though Tsimane' participants with fewer market possessions gave more to out-group members, supporting ACP's hypothesis about strategically building between-group relationships to gain resource access, it may be instead that Tsimane' participants who are more involved in the market economy have more exposure to discrimination at the hands of out-groups and are thus more parochially altruistic \citep{pisor2018diversify}. When asked about collas and cambas, many Tsimane' talked about their access to market resources and their encroachment on and destruction of Tsimane' land \citep{pisor2018diversify}. However, according to many Tsimane' participants, their decision to give more to Tsimane' recipients than to recipients from other pueblos ind\'igenas did not reflect a wish to benefit the Tsimane' at the expense of other groups, per the tenets of parochial altruism; rather, like the Ember\'a in Colombia, the participants reasoned that those groups already had plenty of money, so they wished to give the Tsimane' more for themselves (section \ref{discneed}) \citep{Pisor2020}.
	

\subsection{Conflating two questions}\label{twoquest}
Despite the fact that market participation is increasing for all three populations, none of these populations are interacting with other pueblos ind\'igenas for the first time. Not only were the Tsimane' renowned salt traders before the 20th century, but the Moset\'en traded with lowland groups for tools, medicine, and plants \citep{lathrap1973antiquity, ringhofer2010exploring} and, centuries ago, with the Inca (Quechua) for metal goods \citep{godoy2015natural}. Before the arrival of Columbus, the Quechua and Aymara both had trade networks spanning the Andean high plains, valleys, and foothills, ensuring access to foods from different ecozones \citep{klein2011concise}. Note that the common denominator of these relationships is not necessarily that they cross the boundaries of pueblos ind\'igenas, but that they span distance: for example, highland Aymara residents would often trade with lowland Aymara ``colonists.'' ACP had conflated two orthogonal questions about sociality in her work: whether people respond to a lack of resource access with increased preferences for forming long-distance social relationships, and whether a lack of resource access diminishes parochial altruism.

Why is it important to distinguish study of the evolution of long-distance relationships from the study of between-group relationships in humans? Because both resource competition and resource sharing have been important selection pressures over the course of human evolution, and both have likely affected the psychological adaptations we have for evaluating friend and foe---including both our propensity toward parochialism \citep{moya2015different} and our propensity for strategically building long-distance relationships when they are beneficial \citep{pisor2019evolution, pisorjones2020}. Having long-distance relationships is a means to maintain consistent access to resources, something especially important in humans given our high energy throughput and need for specific nutrients and minerals \citep{pisor2019evolution}. By forming relationships outside their communities, individuals can access resources not locally available---like salt, which, in the Amazon, was heavily concentrated in certain areas \citep{reeve1993regional}---or manage the risk of shortfalls that can strike entire communities, such as floods and droughts \citep{pisor2019evolution, pisorjones2020}. Ethnic boundaries, the group divisions most commonly assumed by the parochial altruism literature (section \ref{onepointone}), can be pronounced both in the contexts of between-group resource competition \citep{choi2007coevolution, bellmoya} \textit{and} between-group sharing, when the efficiency of production may be increased if different ethnic groups focus on different products, favoring specialization and resource exchange \citep{barth1956ecologic, brewer1976ethnocentrism, moya2015different}. Long-distance relationships also span ethnic boundaries; even under conditions where ethnic divisions are marked, if long-distance relationships benefit individuals, cultural institutions may permit inter-group social connections \citep{bollig2010risk}, or intrepid individuals may forge their own \citep{pisor2019evolution, schaub2017threat}.\\
\indent
Distinguishing between these two research foci---the evolution of long-distance relationships and the evolution of between-group relationships---helps to clarify three of most common reasons given for why parochial altruism varies across individuals and groups (section \ref{onepointone}). When the benefits of long-distance social relationships or inter-group specialization exceed the costs, parochial altruism can be overridden \citep[e.g.,][]{bellmoya}. This can result from individual preferences to interact with distant individuals \citep{pisor2019evolution} or out-group members \citep{moya2015different, brewer1976ethnocentrism} or, as noted in section \ref{onepointone}, by cultural institutions enforcing inter-group tolerance \citep{fearon1996explaining, fry2018evolutionary}. Though differences in norms between ethnic groups may increase the costs of coordination between them \citep{bellmoya, habyarimana2007does, mcelreath2003shared}, if between-group relationships are beneficial enough, the norms that increase these costs may be lost \citep{bunce2017interethnic, bunce2018sustainability}. This clarifies diffuse arguments about why individuals develop additional loyalties when exposed to people from other ethnic groups or other countries \citep{brewer1976ethnocentrism, beck2006cosmopolitan, hruschka2013economic, buchan2009globalization, fukuyama2001social, mau2008cosmopolitan, singer2011expanding}: when the benefits to be attained from individuals living elsewhere or in other ethnic groups exceed the costs, group-specific norms and group markers can erode \citep{bunce2018sustainability, moya2015different}  or individuals may invest more time or resources in long-distance relationships \citep{braun1982evolution, pisor2019evolution, minnis1985social, wiessner1982risk}.

Note that the strategic building of long-distance or between-group relationships to gain access to resources not locally available is distinct from the material security hypothesis \citep{hruschka2013economic, hruschka2014impartial}. The material security literature discusses the ``basic needs'' of individuals---including maintaining health and having enough food and money \citep{hruschka2014impartial}---which, in the absence of state support, can often be fulfilled through localized cultural adaptations for managing risk \citep{pisorjones2020}. Between-group or long-distance relationships are often not an efficient way to meet basic needs \citep{minnis1985social}. Instead, long-distance relationships are often forged and maintained when the costs of maintaining them can be paid---difficult if one's basic needs are not being met---and the benefits to be gained through them (usually in terms of non-local resource access) exceed those maintenance costs; the same is true of the gains-to-trade possible through inter-group relationships \citep{bellmoya}.\\
\indent
	As aforementioned, long-distance relationships are especially effective at buffering shortfalls that strike entire communities and providing access to non-local resources \citep{pisor2019evolution, pisorjones2020}. As ACP realized her interest in how individuals improve their non-local resource access was really one of forging \textit{long-distance relationships}, not necessarily \textit{between-group relationships}, she returned to Bolivia in 2017 to investigate this clarified research question. Given how different the Tsimane’ were from the Mosetén and Interculturales in their mobility and market participation---not to mention their game play in 2014---ACP focused on the Intercultural and one of the Moset\'en communities in 2017.
	
	\subsection{The reality of long-distance relationships}\label{distance}
	Though long-distance relationships can be important both for maintaining access to non-local resources and for buffering local shortfalls \citep{pisor2019evolution}, long-distance relationships have not proven relevant for buffering local shortfalls among the Moset\'en and Interculturales \citep{pisorjones2020}. In 2014, both the Moset\'en and Intercultural communities (and much of lowland Bolivia) were hit with severe flooding and landslides, severing roads and cutting power and cell service for more than a month. Floods and landslides are nothing new to these communities: although floods usually occur with less intensity, both the Moset\'en and Interculturales have cultural practices for managing the risk of resource shortfall due to flooding, including raising pigs and chickens that can be slaughtered during hard times and, for the Interculturales, a system of loaning bags of rice to neighbors with expected deferred reciprocity of bags returned the next year. When fallback foods were mostly depleted by the flood in 2014, rather than call on long-distance relationships---which could not be reached, due to lost roads and lack of cell service---the Moset\'en marched down their destroyed road to demand the support of the local government. Both communities eventually used motorized canoes to ferry emergency supplies from the local town. Once the waters receded and roads were repaired, community members who could not absorb the cost of their destroyed crops sought loans from local banks. Three years later, when asked who would help them with a loan during a future hypothetical flood, Moset\'en and Intercultural participants were more likely to name same-community individuals (or even the government) than connections at a distance (Figure \ref{boliviaflood}) \citep{pisorjones2020}. In short, because both communities maintain local institutions for managing risks to their resource access (as well as droughts) and because both found no use for long-distance relationships during the last major flood, long-distance relationships do not appear to buffer shortfalls for the Moset\'en and Interculturales \citep{pisorjones2020}.

\begin{figure}[t]
	\begin{subfigure}{0.5\columnwidth}
		\includegraphics[width=\linewidth]{FloodHistogram.pdf}
	\end{subfigure}%
	\begin{subfigure}{0.5\columnwidth}
		\includegraphics[width=\linewidth]{WorkHistogram.pdf}
	\end{subfigure}
	\caption{{\footnotesize Each participant was asked to name someone who could help them (a) with a loan of 500 \textit{bolivianos} (~\$70, or 8 days' wages) if a flood destroyed their crops, and (b) find ``a good job that pays well''; these counts reflect how far away these named individuals lived in km.}} \label{boliviaflood}
\end{figure}

That said, for both the Moset\'en and Interculturales, long-distance social relationships are crucial for access to resources not locally available \citep{pisorjones2020}. First, there is the question of access to markets. Both the Moset\'en and Interculturales rely on middlemen, often based in the capital city of La Paz (seven hours away by car), to purchase their crops and take them to market. Second, the Moset\'en and Interculturales increasingly participate in migrant labor to supplement their incomes. Extra-community connections are key to finding migrant labor work; when asked who they would contact for a ``good job that pays well,'' 65\% of participants named someone outside their community (Figure \ref{boliviaflood}). Third, market participation increases cash income, which translates into increased mobility. Urban social connections increase the financial and logistic feasibility of mobility. The Moset\'en and Interculturales increasingly have business in La Paz and increasingly send their children to university or job training programs. Individuals report that La Paz residents help them navigate local bureaucracy, from completing government paperwork to enrolling in university. Further, given the high cost of lodging in La Paz, urban connections can provide a cheap or free place to stay. Fourth, urban connections provide access to goods only available in the city. La Paz residents send parcels (\textit{encomiendas}) by bus to rural residents; these parcels can contain anything from bread to cell phones or televisions.

	The Moset\'en and Interculturales maintain these long-distance relationships through a variety of means \citep{pisorjones2020}. Reciprocal exchanges of encomiendas are common: residents of La Paz often request fresh produce like plantains and mandarin oranges, which are expensive in the city. Individuals also send or receive money by bank transfer (\textit{giro}) or Western Union; giros may be used to reimburse someone for an expensive encomienda or to loan money. Semi-reliable cell phone service has been available to the Moset\'en and Interculturales since 2010. While connections can be maintained through short (expensive) phone calls, the advent of data coverage in 2016 has increased the use of platforms like WhatsApp and Facebook Messenger among young people; this is how ACP maintains her long-distance relationships with Moset\'en and Intercultural connections as well. Visitation remains important to relationship maintenance: depending on an individual's means, they may take buses, shared taxis, or their own vehicle to visit long-distance connections, often for several days.
	
	Who are these long-distance connections? Unsurprisingly, some are consanguineal kin (related by blood) or affinal kin (in-laws) \citep{pisorjones2020}. Reciprocal relationships with consanguineal kin can be less difficult to manage (and thus less costly) than reciprocal relationships with unrelated individuals because of convergent fitness interests \citep{hruschka2010}. Important to note, however, is that an individual may selectively invest in relationships with certain consanguineal kin who can generate more benefits for them---for example, with particular siblings that are well-positioned to provide non-local resource access. Siblings were commonly named by Moset\'en and Intercultural participants as non-local connections who could provide information on a good job. Marriage can be strategically used to gain non-local resource access through one's spouse or affines \citep{pisor2019evolution, chapais2009primeval}. Some Moset\'en participants, for example, accuse non-Moset\'en men of marrying Moset\'en women to gain access to Moset\'en tribal lands. Moset\'en and Intercultural Catholics strategically use fictive kinship---namely, godparent relationships---to solidify long-distance connections with individuals they believe are wealthy or influential enough to help them or their children \citep{mintz1950analysis}; teachers, doctors, and middlemen, who are often from La Paz and spend time in both locations, are favorite choices. Long-distance friendships are also forged during periods of temporary migration: for example, during stints of migrant labor, while studying at university or in career programs, or, for men, while completing mandatory military service.\\
\indent	
	Interestingly, with respect to ``who'' these long-distance connections are, participants were often hard-pressed to identify the pueblo ind\'igena of their friends. When ACP asked about friends with whom participants have infrequent contact \citep{pisorjones2020}, she asked participants to identify the pueblo ind\'igena of each. First, participants had a difficult time understanding the question: ACP often cycled through several phrases---pueblo ind\'igena, \textit{descendencia} (descent), \textit{parentezco} (kinship)---before a given participant was able to answer. Second, participants often guessed at the response. Some identified all friends from the lowlands as cambas, even though Indigenous peoples are also from the region; others reasoned that if someone lives in La Paz, they must be Aymara, the dominant pueblo ind\'igena in the city. In short, the pueblo ind\'igena of a long-distance social connection was far less salient to participants than might be expected given the political landscape in Bolivia. See \citet{moya2015different} for a similar example from Per\'u.

\subsection{A failed attempt at distinguishing distance preferences from group preferences}

Given her interest in long-distance relationships, as well as the wealth of observational and self-report evidence that long-distance relationships are important to the Moset\'en and Interculturales, ACP set out to design a task that could potentially decouple participants' preferences for long-distance relationships from their preferences for between-group relationships. Drawing on marketing research, she chose a paired comparison choice-based task \citep{rao2014applied} to assess which traits participants preferred in candidate friends. Though individual-level differences in resource access may predict investment in long-distance relationships, and though participants' perceptions of group-level differences in wealth may affect parochial altruism (section \ref{twoquest}; see also the data from Colombia, section \ref{qual}), ACP was also aware that participants with less wealth may have been more incentivized to keep money for themselves in the 2014--2015 economic game. She reasoned that a task not involving money might reveal preferences for forming new social relationships independently of preferences for giving or keeping money. She presented each participant with two cards representing two candidate friends, each with six different characteristics (Figure Sxx); these included where the candidate friend lived, their pueblo ind\'igena, and their religious affiliation, along with three characteristics with which participants made decisions about money in 2014--2015 \citep{pisor2018diversify}. Participants made 18 sequential decisions between pairs of cards, from which we inferred their preferences for the characteristics of a new friend. See supplementary appendix section 2.1.1 for more details.


 \begin{figure*}[t]
 \centering
\includegraphics[width=5in]{Bolivia_CardChoice_Non-Standardized} % this command will be ignored
\caption{{\footnotesize The odds of picking the right-hand card given that each of the listed characteristics appears on the left (pink) or right (blue). Values shown are posterior means with 90\% credible intervals. If the 90\% credible interval appears to the right of the dashed line, that indicates that a given characteristic increases the odds of picking right; if the 90\% credible interval appears to the left, a given characteristic decreases the odds of picking right. Note that the ''lowest" level for each characteristic is held at zero (that is, part of the intercept); these include candidate friends who live in the same community, who are from the same pueblo ind\'igena or of the same religious affiliation, and who are not good, not trustworthy, and have no money.}
} \label{boliviacards}
\end{figure*}



	Despite including information about a candidate friend's location on each card, participant preferences in the choice task did not reflect the documented prevalence of long-distance relationships in these communities (section \ref{distance}). A candidate friend's pueblo ind\'igena was highly salient to participant preferences, with participants preferring candidate friends from their own pueblo ind\'igena (Figure \ref{boliviacards}; note how participants were more likely to pick the right-hand card when the left-hand card was a candidate friend from a different pueblo ind\'igena, and vice versa). Likewise, participants preferred candidate friends from their same religious group; they also had a slight, though inconsistent, preference for friends from their same community over friends from other places. Why the discrepancy between the ethnographic reality and preferences elicited by the choice task? Some participants reasoned aloud during the decision-making process, providing some insight \citep{bernard2017research}. These participants would often identify one characteristic of the six that stood out to them (``this one is from my church, so I pick him'') and continue to make decisions based on that criterion across all pairs of cards. In other words, even though the characteristics of candidate friends varied across cards, participants stopped reading, and thus attending to, the five characteristics other than the one they selected. Not only did preferences in the choice task not reflect the documented prevalence of long-distance relationships, but they did not reflect preferences elicited by the 2014--2015 economic game. Recall that in the economic game, despite some in-group preference, there was substantial out-group giving. In total, 63 participants both completed the 2017 choice task and played the 2014--2015 economic game. Interestingly, there was no relationship between participants' preferences in the 2014--2015 game and their preferences in the 2017 choice task (see supplement appendix section 2.1.3.2). Perhaps this difference in preferences was due to real changes in preferences over time, possibly related to political climate or changes in material wealth (see supplementary appendix section X); however, it is likely at least partially due to the difference in methodological design.
	
	\subsection{Is parochialism in the method?}
How could these methods---observation, survey, an economic game, and a choice task---produce such discrepant results? While observational data reflect real-world behavior, as do self-report data (assuming participants respond accurately), neither the economic game nor the choice task were designed to capture behavior under real-world constraints; accordingly, we should not expect either to map onto real-world behavior  \citep{Pisor2020, gurven2008collective}. The economic game was designed to bring together strangers (the participants themselves and the recipients in the photos) in first-time interactions that might never occur in the real world, especially for the Tsimane' participants who speak little Spanish and participate less in the market economy than do the Moset\'en or Interculturales. However, the game may have been more real-world than the choice task: the use of money in economic games incentivizes decision-making, such that participants feel their decisions have real-world outcomes \citep{guala2005methodology}. Further, for participants with little wealth who perhaps cannot afford to be generous in the real world, economic games provide an opportunity to give, eliciting participant preferences for giving that may usually masked by real-world constraints (section \ref{qual}) \citep{Pisor2020}. ACP designed the choice task to remove all information about a candidate friend except the six characteristics described, which is not at all like how social partners are chosen in the real world \citep[see ][for a relevant review]{barclay2013strategies}. Without monetary incentives to require that participants ``put their money where their mouth is,'' there was no cost to participants to use any available, and even irrelevant, heuristic to navigate the choice task \citep{Pisor2020, xygalatasreligious}. For a related discussion about why methods that originate in Western academic contexts, like the choice task, may fail in the field, see \citet{hruschka2018learning}.

There are certainly reasons to use methodologies that do not approximate the real world, as we discuss at length in \citet{Pisor2020}; ACP hoped that by removing features of the real-world, such as the information in photos and the potential biases introduced by money, she could better understand participant preferences with minimal intrusion from real-world constraints. However, the choice task in particular deviated so far from the real-world constraints that might guide partner choice for Moset\'en and Intercultural participants, ACP is not convinced that it provided much meaningful insight into participants' preferences for new social relationships.

	How do these results bear on the mixed evidence for parochial altruism that researchers have gathered from around the globe? As evidenced by our comparison of the Tsimane', Moset\'en, and Interculturales, parochial altruism does seem to vary, both across individuals and across ethnic groups. However, different methods used with the same individuals can suggest different degrees of in-group favoritism. Methodological design affects how much parochial altruism we do or do not find, from the way a task is explicitly framed---for example, what researchers do (or do not) tell participants about the task---to its implicit features---for example, whether the choice to give to an in-group member does or does not impact out-group members \citep{hagen2006game, lightner2017, Pisor2020}. In a similar vein to other authors \citep[e.g.,][]{hagen2006game, guala2012reciprocity}, we caution against post-hoc theorizing about the presence or absence of parochial altruism. Instead, we recommend that researchers engage in purposeful experimental design---not only in how much they want their method to reflect the real world, but also in whether they wish to pit the in-group against the out-group, especially as in-group favoritism does not appear to consistently generate costs for out-group members (as demonstrated by the Colombian data; section \ref{q4}) \citep{brewer2006evolutionary, cashdan2001ethnocentrism, hruschka2013economic, purzycki2019identity, schaub2017threat, yamagishi2013behavioral}. See \citet{Pisor2020} for suggestions on how to design economic games to evaluate different research questions.
	
	ACP's economic game was a better example of a purposeful design than the choice task, as the latter did not appear to either elicit the preferences she wished to study nor approximate the real world; for an elegant example of purposeful design in the domain of parochial altruism, see the incorporation of threat into the design of an economic game \citep{schaub2017threat}. Further, evidence from Bolivia and elsewhere suggests that social scientists may need to step away from the assumption that groups function as containers for cooperation \citep{moya2015different}. As we have argued here, distinguishing between different selection pressures that may favor interest or disinterest in strangers---as ACP highlighted, separately investigating the roles of long-distance relationships from the roles of between-group relationships---may help us parse the variability researchers have documented in studies purporting to investigate parochial altruism. If we wish to better understand human sociality, we need to take a step back, hone our hypotheses, and then purposefully design our data collection methods accordingly.



\section{General discussion}
Introduced as a term two decades ago, parochial altruism---generating benefits for in-group members at out-group expense---has had wide-ranging influence on the study of human sociality in the social sciences. If the concept is so influential, however, why does parochial altruism seem to vary so much across individuals, communities, and studies? Institutions may offer one explanation as, when there are net benefits to cooperating with out-group members, this can favor the cultural evolution of norms and rules that further enhance the benefits of between-group interaction \citep{fearon1996explaining, fry2018evolutionary, pisor2019evolution}. Features of our psychology offer another possibility: perhaps because of the benefits they offer, we begin to see members of other ethnic groups as part of our in-group \citep{brewer1976ethnocentrism, beck2006cosmopolitan, buchan2009globalization, fukuyama2001social, hruschka2013economic, mau2008cosmopolitan, singer2011expanding}, or perhaps we can afford the time or resources to care about out-group members when our own needs have been met \citep{hruschka2014impartial, silva2014cooperation}. Yet another explanation stems from data suggesting that in-group cooperation does not always require out-group cost generation \citep{purzycki2019identity, hruschka2013economic, yamagishi2016parochial, brewer2006evolutionary, schaub2017threat, cashdan2001ethnocentrism, Rusch2014}---groups can compete by producing more benefits for their members than other groups do, without directly harming out-groups \citep{waring2015}. Importantly, however, the methods researchers use may be responsible for some of the observed variability in parochial altruism, making it difficult to disambiguate methodological from on-the-ground variation \citep{Pisor2020}.

In this paper, we drew on two case studies from rural South America to explore variation in parochial altruism across individuals and communities, paying special attention to different methods indicated different degrees of parochialism. We found differences in parochial behavior across ethnographic, social network, economic experiment, and choice task data. This is unsurprising, as some of these methods (e.g., ethnographic and social network data) measure real-world behavior, which is subject to numerous constraints, while others (e.g., the economic game played in Bolivia) better measure private preferences \citep{Pisor2020}. Parochial preferences varied across communities, across individuals, and even \emph{within} individuals across time. We review these results in light of the different explanations for on-the-ground variability described above, emphasizing their import for the study of parochial altruism more broadly. We then turn to methods, exploring how research design can generate noise in parochial altruism data. Finally, we close by raising two relevant considerations for researchers studying parochial altruism.

\subsection{Empirical variation in parochial altruism}

In the Introduction, we identified three common explanations from the literature for the variation in parochial altruism. We return to these explanations here, reviewing their potential roles in explaining the data from Colombia and Bolivia, and exploring their relationship to another literature on cooperation: that of need-based transfers [e.g.,][]\citep{peterson1993demand, hooper2015inclusive, aktipis2016cooperation, hao2015need, gervais2017rich, cronk2019managing}.

\textbf{Institutions.} In Colombia and Bolivia, some of the most relevant local institutions are churches. By fostering belief in omniscient deities that favor impartial rule-following, religions can instill norms of more equitable treatment towards individuals from other ethnic groups or of other religious affiliations \citep{purzycki2018evolution, lang2019moralizing}. That said, we saw little evidence that religiosity diminished parochialism in these communities. The exception was in the Intercultural community in Bolivia, where individuals who attended church more frequently were more likely to give money to individuals from other ethnic groups or of a different religious affiliation.

\textbf{Basic needs.} Likewise, impartial treatment of out-group members may arise if individuals have their basic needs met; for example, if government programs buffer shortfalls due to environmental variability or lost jobs, individuals can better afford to share their resources equally with those of different groups or who live at a distance \citep{hruschka2014impartial, silva2014cooperation}. In Colombia, coastal Afrocolombians felt they had more than the Ember\'a and thus engaged in need-based giving; this may also indicate that they felt they had their basic needs met, underscoring a parallel between the basic needs approach and the need-based transfer literature. In Bolivia, however, it was those individuals who thought they had \emph{less} relative to others in their communities---and, for the Tsimane', those who had \emph{fewer} market items---who were more generous toward out-group members.

\textbf{Need-based transfers.} Perceptions of need predicted giving---both in the game and, in the Colombian context, in real-life food and money transfers---regardless of whether the recipient was from the same ethnic group or not. Post-decision interviews from economic games underscored the importance of need-based giving: in Colombia, participants attuned to ethnic group membership as an indicator of need; in Bolivia, the Tsimane' did the same, while the Moset\'en and Interculturales looked for cues of need in recipient photos. Note that it was often co-ethnics (e.g., at the inland Colombian site and among the Tsimane') that were perceived to be the most in need. In some instances, this favoritism does reflect parochial altruism, a conclusion we can draw based on the design of our methods. The Tsimane' gave more to other Tsimane' by taking coins from other ethnic and religious groups---generating benefits for in-group members at the cost of out-group members---justifying this by describing out-group members as resource-rich. In inland Colombia, Afrocolombians and Ember\'a both felt the other group had more resources, creating a context of perceived resource competition. This translated into participants taking more money for themselves from out-group members in the leaving game, suggesting that individuals at the inland site will exploit out-group members to benefit themselves (by definition, an in-group member), but it did not translate into animus in the costly reduction game.

\textbf{Exposure.} Is exposure to out-group individuals required to override parochial tendencies and arrive at need-based giving? For the Tsimane', this may be the case. Need-based transfers represent a form of risk pooling \citep{cronk2019managing}. Risk pooling clearly occurs between ethnic groups among the Moset\'en and Interculturales, where delayed reciprocity among same-community individuals helps households manage food shortages during droughts \citep{pisorjones2020}. Given this, it is perhaps unsurprising that Moset\'en and Intercultural participants gave money to members of other ethnic groups in the economic game: they are both more frequently in contact with members of other ethnic groups and regularly engage in risk pooling with them. This raises the question of whether ''out-group" is even an appropriate designation for individuals from other ethnic groups; after all, Moset\'en and Intercultural participants were hard-pressed to identify the ethnic background of their friends, suggesting that ethnic boundaries in this context do not act as containers for cooperation [e.g.,][] \citep{brewer1976ethnocentrism, moya2015different}. That said, among the Tsimane' individuals interviewed, risk pooling rarely involves members of other ethnic groups (see also \citet{jaeggi2016reciprocal}), partially because they do not live nearby---which itself partially reflects Tsimane' decisions about where to live. If the Tsimane' had easier access to members of other ethnic groups, via cheaper, easier transportation and better cellular service, would they build more relationships spanning ethnic boundaries? Ethnographic and self-report data suggest they would not: the Tsimane' have long experienced discrimination from members of other ethnic groups and view themselves as the ''have-nots" and other ethnic groups as ''haves."

In short, while the institutional, basic needs, and exposure explanations for variation in parochial altruism may play some role in rural Colombia and Bolivia, the importance of need-based transfers, perhaps to build and maintain the relationships required for effective risk management, has more direct explanatory power. Further, the role of long-distance relationships in risk management should not be overlooked, particularly as these relationships can co-occur with parochial altruism---the bulk of an individual's social ties will probably still be to in-group individuals \citep{bollig1993intra, brewer1976ethnocentrism, lathrap1973antiquity, bowles2004persistent}. Institutions can further augment the net benefits to be gained through long-distance relationships, increasing their frequency \citep{pisor2019evolution}. In sum, looking to contexts outside of rural Colombia and Bolivia, each of these sources of variation---institutions, basic needs, exposure, and risk management---may play a larger or a smaller role, altering social ties and the resources they transfer accordingly.

\subsection{Method-based variation in parochial altruism}

Is there variation in observed parochial altruism that is not due to variation in parochialism on the ground? This was the primary focus of the present paper, and our review of the evidence strongly suggests that at least part of the variation documented across various studies of parochial altruism is due to method chosen. Here, we highlight (1) the role of experimental design---in this case, (a) the relevance of incentivizing decision-making and (b) whether participants can engage in in-group favoritism without automatically generating costs for out-group members---and (2) the distinction between measuring real-world vs. private preferences.

\textbf{Incentivized decision-making vs. ''cheap talk."} A common trope in anthropology is that participants can tell researchers whatever they like---talk is cheap. Offering participants incentives, like money in economic games, encourages them to put their ''money where their mouth is" \citep{xygalatasreligious, gurven2008collective, Pisor2020}. However, there is a limitation to incentivizing decision-making: where wealth is unequal, the value of incentives varies across individuals such that those who need money may do whatever it takes to get it and keep it. In Colombia, participants' decisions to transfer more to in-group vs out-group members was independent of their wealth, suggesting that the incentive worked. By providing a small windfall of money, in fact, CTR lifted the constraints of the real world, allowing participants to give as they wished, thus revealing their private preferences (see \citet{Pisor2020} for further discussion). Likewise, participants who saw themselves as having less than others in their community gave \emph{more} in Bolivia, suggesting that the game captured participant preferences rather than their finances. However, when ACP attempted to steer clear of incentivized decision-making by using a choice task, talk became cheap: participants often used simple in-group vs out-group heuristics to choose candidate friends, inconsistent with their behavior in the economic game, the lack of salience of ethnic group membership in the real world, and the friendships they maintain that span both distance and ethnic boundaries. In sum, the degree to which participants espouse parochialism in a study may partially reflect whether they are motivated to respond.

\textbf{Decoupling in-group favoritism from out-group cost generation.} Our data suggest that the link between in-group favoritism and costs for out-group members---whether as a byproduct of generating benefits for in-group members or, in its more extreme form, as spite or animus---can be decoupled \citep[see also][]{cashdan2001ethnocentrism, hruschka2013economic, schaub2017threat, yamagishi2016parochial}. While in the allocation and the Bolivian games, in-group favoritism comes at the price of negative externalities for out-group members (coins given to in-group members mean coins \emph{not} given to out-group members), the leaving game provides enough coins for \emph{all} recipients to receive one. Participants could thus leave money for in-group members at no cost to out-group members and vice versa (although note that the participants themselves received fewer coins if they left more). The leaving game data mirror the real-world data: at the coastal site, Afrocolombians were no more likely to leave coins for, or give food or money to Afrocolombians than Ember\'a; at the inland site, both groups preferentially left coins for and gave food or money to in-group individuals. In short, when experimental methods do not explicitly pit in-group against out-group members, in-group favoritism and out-group cost generation can be decoupled.

\textbf{The real and the private worlds.} In another paper, we discuss the distinction between eliciting real-world vs private-world preferences at length---that is, what participants prefer to do given the constraints posed by their financial circumstances, the expectations of others, the norms of their community, etc., versus how they would prefer to behave if these constraints were minimized \citep{Pisor2020} (see also \citet{Naar2020}). Here, we showcased methods that captured both. RICH games were designed to provide insight into real-world relationships \citep{gervais2017rich, Pisor2020}, and in Colombia they seem to: social networks of food and money transfers are, by definition, real world, and they largely mirror participants' game play. However, participants sometimes indicated they wished they had more to give outside the game context, indicating their private preferences---what they would do if they could. Focused on participants' \emph{interest} in between-group relationships, rather than their realizations, ACP designed a game and a choice task to measure private-world preferences in Bolivia. While the choice task was too unmoored from reality, the game did seem to elicit preferences, as participants often discussed need in post-game interviews. In short, researchers who wish to study parochial altruism, or other aspects of human social relationships, should reflect on whether they wish to measure the psychology of the phenomenon or its real-world consequences. Different methods will be required to study each.

\subsection{How can we more accurately measure variability in parochial altruism?}

How can researchers interested in parochial altruism minimize the noise in their data due to their chosen methods? We recommend two simple considerations to guide methodological design, considerations which apply with equal force to other topics of research.

\textbf{Consideration 1: Be deliberate in methodological design.} If at all possible, we recommend that researchers select and design their methods to match their research question, rather than use a common method (e.g., classical economic games) and theorize about the results \emph{post hoc} \citep{Pisor2020}. For example, as part of their design, allocating money to in-group members in the RICH allocation game or the Bolivian game automatically generates benefits for out-group members, who then will not receive coins; on the other hand, the leaving game allows participants to generate benefits for both. To take another example, questions about individuals' social networks provide insight into in-group favoritism at out-group expense \emph{given} real-world constraints; decision-making in experiments that minimize these constraints can provide insight into the underlying psychological mechanisms that guide these real-world outcomes. See \citet{schaub2017threat} and \citet{yamagishi2016parochial} for examples of the question-then-method approach.

\textbf{Consideration 2: Triangulate.} It is unlikely that we will build any deep understanding of parochial altruism, or any social phenomena, just by picking one game, choice task, or other methodological approach. For example, if we had not integrated game with self-report data in Colombia, we would potentially miss the relevance of need-based transfers in observations of in-group favoritism or lack thereof. Likewise, if we had relied only on the choice task in Bolivia, we would conclude that the Moset\'en and Interculturales are parochial, even though ethnographic data reveal that ethnic group membership has limited salience in daily life. In sum, researchers should use multiple data sources to triangulate the reality of parochial altruism on the ground \citep{Friedman2004, Pisor2020, Naar2020, gurven2008collective}. For an example of the power of triangulation in studying parochial altruism, see XXX.

%In this paper, we have further seen that norms for need-based giving can be powerful drivers of cooperative transfers in both of our field-sites, mirroring similar findings in a wide array of other  communities . If members of an underprivileged group view themselves as more in-need, then they be more likely to direct aid to co-ethnics and exploit the out-group: in such cases ethnicity \textit{per se} may mediate the effect of need on the probability of transfers, without being the causal driver of such transfers.  The data from both the Ember\'a and  Tsimane' case studies support such an interpretation. The Afrocolombian data also square with this interpretation: at the inland field site, Afrocolombians perceive the Ember\'a community to be comparably well-off and thus direct aid to other Afrocolombians and exploit the Ember\'a in the exploitation game, but on the coastal site Afrocolombians do not exploit the out-group, as they are perceived to be more in-need.

%Need-based giving has been documented in a wide array of other communities \citep[e.g., ][]{peterson1993demand, hooper2015inclusive, aktipis2016cooperation, hao2015need, gervais2017rich, cronk2019managing}. 

%Sociologists have taken a keen interest in understanding the factors that increase social cohesion in multiethnic societies, and have focused especially on whether prosocial behavior extends beyond closed close-knit networks and in-group boundaries \citep{Baldassarri1183}. They highlight two features of modern societies---social differentiation and economic interdependence---that may increase constructive interactions with out-group community members \citep{Baldassarri1183}. More broadly, relationships spanning group boundaries---or sometimes, more importantly, spanning distance, such that shortfalls are less likely to be correlated \citep{wiessner1982risk, cashdan1983territoriality, spielmann1986interdependence, smith1988risk, junker1996hunter, fitzhugh2011modeling}---are so important that cultural institutions are often created or re-purposed to lower the transaction costs for establishing such connections: e.g., exogamy \citep{chapais2009primeval}, institutional  support for long distance trade \citep{michalopoulos2012trade}, norms of hospitality \citep{selwyn2000anthropology}, ritualized relationships \citep{malinowski1922argonauts, hruschka2010friendship}, and fictive kinship \citep{hruschka2010friendship, halbich2010ritual, wiessner1982risk}. So even if cognitive adaptations for in-group favoritism and out-group animus are an important part of a pan-human psychology, so are mechanisms and institutions that moderate such biases. It is possible to see both between-group tensions and between-group dependence coexisting \citep{bollig1993intra, brewer1976ethnocentrism, lathrap1973antiquity, bowles2004persistent}. 

%Broader explanations for in-group preferences in the absence of out-group animus also abound:  ethically-structured social networks need not arise from preferences for in-group favoritism at out-group expense. For example, socio-political homophily may lead to ethnically structured social networks, even when individuals express no preference for within-ethnicity interactions \textit{per se } \citep{stark2012double}. Likewise, models based on  coordination games suggest that ethnically structured populations may emerge simply from fewer miscoordination errors when engaging in within-ethnicity interactions \citep{mcelreath2003shared, bunce2017interethnic, bunce2018sustainability}, and that cross-cultural competence can mitigate such miscoordination errors and be selected for when there are benefits to between-group interactions \citep{RePEc:osf:socarx:bwtvu}. In a similar vein, others have suggested that social assortment may emerge because there are fewer transaction costs within ethnolinguistic \citep{habyarimana2007does} or religious groups \citep{ensminger1997transaction}.  Because population size,  resource control, and short-side power are frequently changing over time, incentives for cross-cultural competence and relationships spanning group boundaries are likely to be in flux as well.  

\section{Conclusions}

Parochial altruism, often assumed to be univerally present in humans, is in fact universally variable. Researchers have suggested several pathways that might generate this variation: institutions and having one's basic needs met may lower the costs of cooperating with out-group members---where ''out-group" usually refers to members of another ethnic group---while exposure to out-group members can increase the perceived benefits of cooperation. However, another source of measured variation in parochial altruism is likely the methods used by researchers. In the present paper, we reviewed two case studies, one from rural Colombia and the other from rural Bolivia, to illustrate the role of methods chosen in conclusions one might draw about the relative presence or absence of parochial altruism in a given sample. We provided an example of the consequences of a poorly designed method of data collection, demonstrated the merits of triangulating levels of parochial altruism using multiple methods, and cautioned researchers not to conflate research questions---for example, confusing questions about between-group vs long-distance relationships. We then made concrete suggestions about how researchers can better design their studies to minimize observed variation in parochial altruism due to method chosen. In closing, a word of caution: the possibility that existing observations of parochial altruism (or lack thereof) are partially a product of the method used could have large implications for how we think about human sociality and its flexibility.

\section*{Acknowledgments}
ACP: Thanks to Daniel Hruschka for his contributions to model design and to study design for 2017 data collection, to Michael Gurven for his collaboration on study design for 2014-15 data collection, to Cristina Moya for suggesting the choice task, and to the Max Planck Institute for Evolutionary Anthropology Department of Human Behavior Ecology and Culture for helpful discussion. Thanks to research assistants Bernabe Nate Añes, Jacinta Álvarez, and Amira Siquimen for their contributions to data collection, and to the Tsimane' Health and Life History Project for logistical support. Thanks to the Max Planck Institute for Evolutionary Anthropology Department of Human Behavior Ecology and Culture, the US National Science Foundation (DDRIG 1357209), the Wenner-Gren Foundation (Dissertation Fieldwork Grant 8913), the UCSB Broom Center, and UCSB Anthropology for funding.

CR: Thanks to the Max Planck Institute for Evolutionary Anthropology, Department of Human Behavior Ecology and Culture for funding and helpful discussion. 


%%%%%%%%%% Insert bibliography here %%%%%%%%%%%%%%
\bibliographystyle{plainnat}
\bibliography{main}



\end{document}



